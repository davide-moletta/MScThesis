\chapter{Conclusions}
\label{cha:conclusions}

This thesis discussed the design, development, and analysis of \textit{STEERED: a
Secure Trusted Execution Environment for RISC-V-based Embedded Devices} which poses
itself as a significant contribution to enhancing security in resource-constrained
environments. The project successfully addresses key challenges related to
embedded device security, including mitigating control-flow hijacking attacks such
as Return-Oriented Programming and Jump-Oriented Programming. With the
integration of advanced mechanisms like a Shadow Stack and Control Flow Graph, \textit{STEERED}
ensures robust enforcement of Control Flow Integrity, a critical step in preventing
unauthorized execution flows or arbitrary code execution.

In chapter \ref{cha:project} we have seen the process behind \textit{STEERED}'s
design and implementation highlighting its key features. We have also seen how it
adheres to the principles of lightweight and efficient security, making it
suitable for devices with limited computational resources. The system leverages the
flexibility and extensibility of the \textit{RISC-V} Instruction Set
Architecture to create a tailored solution that balances security and
performance. The proposed infrastructure achieves this balance through meticulous
code instrumentation, optimized Physical Memory Protection configurations, and efficient
trap management mechanisms. Additionally, the modular design of \textit{STEERED}
ensures that it remains adaptable to various use cases and allows for further
customization. Lastly, with the Proof of Concept presented in section
\ref{sec:project_poc} we demonstrated \textit{STEERED}'s ability to effectively detect
and mitigate attacks that aim at disrupting the code execution flow.

Chapter \ref{cha:ta} was focused on providing an in-depth view of the threat landscape
with the threat model. Additionally, we proved why and how \textit{STEERED} can
effectively protect the code from such threats by showcasing the utilized
testing methodologies. Also, we discussed the weak points of the project, describing
its limitations in protecting from specific attacks and discussing potential enhancements
to provide a higher degree of security.

A comprehensive performance analysis presented in chapter \ref{cha:pa} illustrates
the practical viability of incorporating \textit{STEERED} into real-world
applications while maintaining a high level of system efficiency. The analysis delves
into the average overhead associated with the implementation of \textit{STEERED},
highlighting the importance of this metric in evaluating its effectiveness. We
explored various optimization techniques specifically designed to mitigate
elevated overheads that may arise in certain unusual scenarios. By addressing these
potential challenges, we aim to ensure that \textit{STEERED} can be seamlessly integrated
into existing systems, ultimately enhancing overall performance while minimizing
any adverse effects.

The purpose of chapter \ref{cha:rtos} was to provide an implementation idea to
integrate the system with widely used Real-Time Operating Systems such as
\textit{FreeRTOS} and \textit{Zephyr RTOS}. Such integration could enhance the project's
versatility and readiness for broader adoption, allowing for the execution of
more complex tasks. We focused on this implementation because many real-world applications
are highly dependent on the features provided by RTOSes to carry out their tasks.

Lastly, the thesis also identifies several areas for future exploration, including
the incorporation of additional security mechanisms, further code optimization, and
exhaustive testing under diverse operational conditions in chapter \ref{cha:future}.
These potential enhancements aim to refine the system and broaden its
applicability, extending its impact on the security of embedded systems in the rapidly
expanding Internet of Things landscape.

In conclusion, the framework known as \textit{STEERED} marks a significant milestone
in the realm of embedded system security. This innovative approach introduces
practical solutions specifically designed to protect \textit{RISC-V} microcontrollers
from a variety of emerging cyber threats that have been increasingly targeting these
systems. By addressing not only the immediate security challenges faced by these
microcontrollers but also laying the groundwork for future advancements, \textit{STEERED}
represents a forward-thinking initiative in cybersecurity.

A key achievement of \textit{STEERED} is its focus on safeguarding untrusted code
from control-flow tampering attacks, a common and dangerous threat in the
embedded systems landscape. The infrastructure developed under \textit{STEERED}
has been meticulously designed to ensure that security measures do not
detrimentally affect the performance of resource-constrained systems. This balance
between security and efficiency is critical, as many embedded devices operate
under stringent resource limitations.

Furthermore, this work highlights the increasing necessity of secure and efficient
design principles within embedded device infrastructures. As embedded systems
continue to integrate deeper into various aspects of modern technology
ecosystems, ranging from consumer electronics to critical infrastructure, the
importance of robust security mechanisms cannot be overstated. The findings and advancements
brought forth by \textit{STEERED} not only enhance the immediate security
posture of \textit{RISC-V} microcontrollers but also serve as a foundational framework
for ongoing innovation in the field, ultimately contributing to a safer and more
secure technological landscape.