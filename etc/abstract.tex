\chapter*{Abstract}
\label{cha:abtract}
\addcontentsline{toc}{chapter}{Abstract}

The proliferation of embedded devices across industries and daily life, from consumer
electronics to healthcare and industrial automation, has amplified the urgency
of securing these systems against an escalating array of cyber threats. This thesis
presents a novel security framework designed to enforce Control Flow Integrity to
protect embedded systems based on the open \textit{RISC-V} Instruction Set
Architecture. Unlike proprietary ISAs, \textit{RISC-V} enables the development of
customized processors and solutions tailored to specific use cases. This project
exploits this flexibility to address control-flow hijacking attacks, such as Return-Oriented
Programming and Jump-Oriented Programming, which threaten the integrity of
execution paths in microcontroller-based devices.

The proposed solution incorporates lightweight Control Flow Integrity
enforcement mechanisms, including a shadow stack and a Control Flow Graph, to
ensure that software execution adheres strictly to predefined paths. This prevents
unauthorized deviations caused by potential attackers, ensuring the security of
both the system and its data. The solution employs code instrumentation techniques
that automatically integrate CFI into user applications, simplifying deployment
and reducing development time. Moreover, the use of \textit{RISC-V}'s Physical Memory
Protection safeguards critical data structures, enhancing the robustness of the
framework.

The design philosophy behind the project emphasizes efficiency and scalability, enabling
it to operate effectively on resource-constrained embedded devices. A key
feature is its ability to provide strong security guarantees with minimal
performance overhead, making it suitable for IoT devices that often have limited
computational and power resources. A Proof-of-Concept implementation
demonstrates the effectiveness of our solution in defending against control-flow
hijacking attacks. Detailed performance analysis reveals that the framework achieves
a balance between security and operational efficiency, with an evaluation of
memory overhead and execution time supporting its feasibility for practical
applications.

Furthermore, this thesis explores the integration of Real-Time Operating Systems
such as \textit{FreeRTOS} and \textit{Zephyr RTOS}, showcasing the project's
adaptability to complex software environments. This integration broadens the applicability,
enabling the secure execution of real-time and multitasking applications in IoT
and embedded domains.

In conclusion, the proposed solution addresses the pressing challenge of securing
embedded \textit{RISC-V} devices by combining innovative security mechanisms, automated
processes, and adherence to \textit{RISC-V}'s open and customizable principles.
This work contributes significantly to the field of embedded system security by
offering a scalable, efficient, and customizable framework that meets the needs
of modern IoT environments while anticipating future challenges in the
cybersecurity landscape.
