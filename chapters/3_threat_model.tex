\chapter{Threat Model and Assumptions}
\label{cha:threatmodel}

This chapter aims to provide a detailed exploration of the threat model and the
underlying assumptions associated with the project. By analyzing potential risks
and vulnerabilities, we seek to understand the implications these threats may
pose to the project's success and integrity.

During the design and development phase of the project, we focused on a \textit{RISC-V}-based
embedded device that operates within a network environment compromised by an
attacker. The device is running unverified code in a bare-metal environment,
meaning there is no underlying operating system to manage resource allocation or
security. This setup may expose the device to various risks. One or more
vulnerabilities that could be exploited by malicious actors exist within the
executable. Additionally, there is a possibility that the codebase may contain
malicious software or gadgets.

The threat is posed by a malicious actor who, thanks to any present vulnerability
can hijack the device by perpetrating one or more control flow hijacking attacks
among the ones seen in section \ref{sec:background_cfa}. Note that given the
bare-metal execution environment, there are no dynamically linked or shared libraries,
so we focus mostly on attacks such as Return-Oriented Programming and Jump-Oriented
Programming. Also, the attacker may have access to the source code or binary
file. As the project aims to detect and prevent control flow hijacking attacks,
other attack types are not within its scope, and alternative approaches should be
utilized to address them.

The target device in question is based on the \textit{RISC-V} Instruction Set
Architecture, which does not impose any particular assumptions regarding its specifications
or configurations. It is important to note that we are operating under the
assumption that, despite the attacker possibly having access to the source code of
the software, they are unable to modify this source code in any way that would
allow them to upload or flash a version that has been tampered with onto the
device. Furthermore, we maintain the assumption that the attacker cannot
physically interfere with or modify the target device in any manner. This includes
actions such as opening the device, accessing hardware components, or conducting
hardware-based attacks that could compromise the system's integrity.
