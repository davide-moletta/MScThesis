\chapter{Threat Model and Assumptions}
\label{cha:threatmodel}

This chapter aims to provide a detailed exploration of the threat model and the
underlying assumptions associated with the project. By analyzing potential risks
and vulnerabilities, we seek to understand the implications these threats may
pose to the project's success and integrity.

During the design and development phase of this project, we focused on a \textit{RISC-V}-based
embedded device that operates within a network environment potentially
compromised by an attacker. The device is executing unverified code in a bare-metal
environment, meaning there is no underlying operating system to manage resource
allocation or security. This setup may expose the device to various risks. At
least one vulnerability is present within the executable, which could be exploited
by malicious entities. Additionally, there is a possibility that the codebase may
contain malicious software or gadgets.

The threat is posed by a malicious actor whom, thanks to any present
vulnerability can hijack the device by perpetrating one or more control flow
hijacking attacks among the ones seen in section \ref{sec:background_cfa}. Note that
given the bare-metal execution environment there are no dynamically linked or shared
libraries so we mostly focus on attacks such as Return-Oriented Programming and Jump-Oriented
Programming. Also, the attacker has undefined access to the source code or binary
file, meaning that he may or may not have access to them. Since the focus of the
project is to detect and mitigate control flow hijacking attacks, any other type
of attack is out of scope and should be addressed via other means.

The target device in question is fundamentally based on the \textit{RISC-V} Instruction
Set Architecture, which does not impose any particular assumptions regarding its
specifications or configurations. It is important to note that we are operating under
the assumption that, despite the attacker possibly having access to the source
code of the software, they are unable to modify this source code in any way that
would allow them to upload or flash a version that has been tampered with onto the
device. Furthermore, we maintain the assumption that the attacker does not have the
capability to physically interfere with, or modify, the target device in any manner.
This includes actions such as opening the device, accessing hardware components,
or conducting any form of hardware-based attacks that could compromise the
integrity of the system.
