\chapter{Future Works}
\label{cha:future}

In this thesis, we have seen that \textit{STEERED} poses itself as a
foundational approach to improving the security of \textit{RISC-V}-based embedded
systems. Moreover, it showed promising results in both the threat and performance
analysis, making it a suitable choice for real-world applications. However,
further advancements can enhance its applicability, efficiency, and robustness. This
chapter outlines potential directions for future work to expand the capabilities
of \textit{STEERED}. These future directions aim to establish \textit{STEERED}
as a cornerstone in the development of secure and efficient embedded systems. Addressing
these challenges will ensure its relevance and applicability in an ever-evolving
cybersecurity landscape.

\section{Additional Security Mechanisms}
\label{sec:future_security}

In previous chapters we discussed and demonstrated the ability of \textit{STEERED}
to detect and prevent control-flow attacks such as Return-Oriented Programming or
Jump-Oriented Programming. However, in security demanding environments it may be
necessary to introducer further techniques to protect the Control Flow Integrity
of the code. Here we list some security mechanisms that we plan to integrate in
\textit{STEERED}.

A \textit{Stack Canary} is a security mechanism used to detect and prevent stack-based
buffer overflow attacks, a common type of vulnerability in programs. It involves
placing a small, random value (the ``canary'') in memory just before the stack's
return address. This value acts as a sentinel and is checked for integrity
before the function returns control to its caller. If a buffer overflow occurs,
it overwrites memory beyond the intended bounds, which could potentially overwrite
the canary value before reaching the return address. Before the function returns,
the program compares the current canary value with the original value. If the
values differ, the program detects the overflow and terminates the execution to prevent
further exploitation. \textit{Stack Canaries} are simple yet effective as they provide
a straightforward way to detect stack corruption caused by buffer overflows,
also they are widely supported by many compilers. However, the process of
placing and checking canaries adds slight computational and memory overhead. Moreover,
\textit{Stack Canaries} only protect against stack-based buffer overflows,
leaving other memory regions vulnerable. In conclusion, we think that \textit{Stack
Canaries} would be a good addition to \textit{STEERED} as they would be easy to
implement and they provide a slight increase in security.

Since \textit{STEERED} suffers from source code modification as we have no way of
telling if the code was modified or not and this could affect the correctness of
forward and backward edge controls we could add an hashing function to the project.
With this function we could take the source code of \textit{STEERED} and generate
an hash. Note that we plan to take into account only the ``static'' part of
\textit{STEERED} and not the code imported by the user. With this we could
perform a check at each compilation by simply comparing the initial hash with
the freshly generated one. If the hash differs it means that the source code has
been tampered with and should not be trusted. This would lead to a simple and
fast way to determine if \textit{STEERED} is secure or if it has been modified.

Moreover, many device-specific security measures could be explored as there may
be some devices that provides hardware modules which could be used to further
increase the security of the project. However, this process requires exhaustive
testing and would lead to possible enhancement only on few embedded devices.

Note that we can't add security measures such as Data Execution Prevention (DEP)
or Address Space Layout Randomization (ASLR) to the binary as they would
conflict with \textit{STEERED}'s configuration. Data Execution Prevention
requires to set memory regions with either write but not execute privileges or
execute but not write privileges. This would obviously impact \textit{STEERED}
as it provides memory regions that requires both privileges. Address Space Layout
Randomization instead would randomize all the addresses so that they are different
at each execution. While this technique provides good security it would render
the Control Flow Graph completely useless as the addresses we collect during the
extraction would be different from the addresses at execution time leading to a
situation where each forward edge control fails.

Lastly, note that \textit{STEERED} is designed to protect the code from control-flow
attacks and must be paired with other security measures before deployment. This is
needed in order to ensure security on every attack surface. Otherwise, a threat actor
may be able to exploit other vulnerabilities to perpetrate an attack bypassing \textit{STEERED}'s
security features.

\section{Code Optimization}
\label{sec:future_optimization}

Although \textit{STEERED} has been designed with performance considerations, further
optimizations could enhance its usability in a broader range of embedded applications.
During the Performance Analysis in chapter \ref{cha:pa} we have seen that
\textit{STEERED} comes with an acceptable average time and space overhead.
However, there are edge cases where this is not true and the overhead grows exponentially.
In future works we plan to provide optimization solutions to cover such cases
lowering the time and space impact of \textit{STEERED} on the execution.

Firstly, we propose a solution to address the space overhead of deeply recursive
algorithms. We have seen that in these cases we need to allocate a very big Shadow
Stack which increases the size of the produced binary. To solve this, we propose
to add a \textit{peek} function which allows to look at the first element of the
Shadow Stack without removing it. When we need to perform a forward edge control
and consequently push the return address into the Shadow Stack we first check the
top value. If the addresses are the same we avoid pushing the same value more
times. On the other hand, when we need to perform a backward edge control we
look at the first value of the Shadow Stack. If the two addresses are equal we approve
the return instruction without popping the value. Instead, if the addresses differ,
we pop two values and compare the second one with the return address we are
checking. With this solution we could effectively address the problem with deeply
recursive algorithms without affecting the security capabilities of \textit{STEERED}.

Secondly, we provided a solution to reduce memory usage in cases where the user code
is small. For example, say that the user code starts at address $0x4038A000$ and
ends at address $0x4038F000$. Well, in this case we can ignore the first $16$ bits
of the address as they do not provide useful information. This means that we
could modify the Shadow Stack to store \textit{unsigned short} values instead of
\textit{unsigned int} values. Such a solution would halve the memory requirement
for the Shadow Stack. Although this solution is very effective in memory
management it is only applicable when the codebase is very small.

Moreover, we have seen that if the user-imported code is very big, take as an
example a firmware, the Control Flow Graph would drastically increase in size.
This is because, the larger the code, the larger is the possibility to have many
jump instruction from different source addresses to different destination
addresses. To address this problem, we propose to use an Hash Map instead of a
two dimensional array to represent the CFG. Although this solution does not reduce
memory usage, it allows for faster lookups as Hash Maps provide accesses in
constant time ($\mathcal{O}(1)$).

Lastly, converting the source code from \textit{C} language to pure \textit{Assembly}
language could increase performances and lead to further optimization solutions to
reduce time and memory overhead.

\section{Exhaustive Testing}
\label{sec:future_testing}

During the Threat Analysis in chapter \ref{cha:ta} we showcased the
effectiveness of \textit{STEERED} in protecting the Control Flow Integrity of the
user code. Moreover, we proved through tests that attempts to perform an unauthorized
control transfer are immediately detected and prevented by the Control Flow Integrity
Enforcer of \textit{STEERED}. However, we also pointed out that the presented
project is not able to mitigate non-control-flow attacks. In future works we plan
to perform exhaustive testing of the project to further inspect the behavior of \textit{STEERED}
in diverse environments. This could lead to either a stronger proof that \textit{STEERED}
is actually able to protect the code or to the discovery of untested scenarios. In
case of the latter we plan to provide a solution to eventual unexpected behavior.

\section{RTOS Implementation}
\label{sec:future_rtos}

The main focus for future works sees the integration of a Real-Time Operating
System such as \textit{FreeRTOS} or \textit{Zephyr RTOS} inside \textit{STEERED}.
Such task is of great importance due to the fact that an implementation
providing the features of both an RTOS and \textit{STEERED} would drastically increase
the number of fields in which this project could be used. In fact, this would open
door to more complex solution and would allow embedded devices to carry out complex
task within the secure execution environment provided by \textit{STEERED}.