\chapter*{Scaletta}
\label{cha:Scaletta}

\begin{itemize}
  \item Introduction

  \item RISC-V
    \begin{itemize}
      \item Specifics

      \item Extensions

      \item Registers \& CSRs

      \item Privilege levels

      \item Interrupts and traps

      \item PMP
    \end{itemize}

  \item Project
    \begin{itemize}
      \item Formalization

      \item Specifics

      \item Interrupts Service Routine

      \item Shadow Stack

      \item PMP

      \item controls

      \item PoC

      \item PoW
    \end{itemize}

  \item SBI Implementation

  \item FreeRTOS Implementation

  \item Future works
    \begin{itemize}
      \item Forward edge controls

      \item SBI Implementation

      \item FreeRTOS Implementation
    \end{itemize}

  \item Conclusions
\end{itemize}

\begin{itemize}
  \item 1. Introduction: Introduction to the RISC-V architecture and why it's
    important in modern embedded systems. Discuss the motivation for the project,
    such as enhancing security with Control Flow Integrity mechanisms like
    shadow stacks. Define the goals of the project, including implementing a
    shadow stack and analyzing the performance and security of RISC-V. Brief overview
    of the thesis structure.

  \item 2. RISC-V Architecture Overview
    \begin{itemize}
      \item 2.1 Architecture Specifics: A high-level overview of the RISC-V
        Instruction Set Architecture (ISA), its open-source nature, modularity, and
        scalability.

      \item 2.2 Instruction Formats

      \item 2.3 Extensions: Explanation of common RISC-V extensions (e.g., Atomic
        instructions, Compressed instructions, Floating-point extensions) and
        how they enhance the architecture.

      \item 2.4 Registers and CSRs: Detailed description of RISC-V registers and
        Control Status Registers (CSRs).

      \item 2.5 Privilege Levels: Discussion on the privilege levels in RISC-V (User,
        Supervisor, Machine Mode) and their role in system security.

      \item 2.6 Interrupts and Traps: Description of RISC-V's interrupt handling
        mechanism, traps, and the role of the Machine Trap-Vector Base-Address
        Register (MTVEC).

      \item 2.7 Physical Memory Protection (PMP): Explain PMP and how it
        controls access to memory regions for enforcing security policies.

      \item 2.8 RISC-V Toolchain
    \end{itemize}

  \item 3. Project Implementation
    \begin{itemize}
      \item 3.1 Formalization of the Problem: Define the core problem that the
        project addresses (Control Flow Integrity through shadow stacks). Discuss
        the relevance of preventing unauthorized address jumps.

      \item 3.2 Project Specifics: Overview of how the project works, including initialization
        and setup of the RISC-V environment for shadow stack implementation.

      \item 3.3 Interrupt Service Routine (ISR): Detailed description of the ISR
        implementation, how interrupts are handled, and their integration with
        the shadow stack.

      \item 3.4 Shadow Stack Implementation: Explain how the shadow stack is
        implemented, how it maintains control flow integrity, and its
        relationship with jump and return instructions.

      \item 3.5 PMP Configuration: Explanation of the PMP settings used to
        protect the shadow stack from unauthorized memory access.

      \item 3.6 Security Controls: Describe the forward-edge and return-address
        controls, Control Flow Graph (CFG) matching, and how these mechanisms
        prevent exploitation.

      \item 3.7 Instrumentation

      \item 3.8 Proof of Concept (PoC): Present the prototype, testing methodologies,
        and results that demonstrate the successful implementation of the
        project.

      \item 3.9 Proof of Work (PoW): Discuss the challenges encountered and the
        approaches taken to overcome them during the implementation.
    \end{itemize}

  \item 4. Threat analysis: Introduce a short section where you outline the
    types of security threats you’re defending against (e.g., return-oriented programming
    (ROP), buffer overflows) and how the shadow stack specifically counters them.

  \item 5. SBI (Supervisor Binary Interface) Implementation - Discuss the implementation
    of the SBI in the project and how it handles system calls and traps in the RISC-V
    environment.

  \item 6. FreeRTOS Implementation - Explain the integration of the project with
    FreeRTOS, focusing on task scheduling, memory protection, and how the shadow
    stack operates within a real-time operating system.

  \item 7. Future Work and Enhancements
    \begin{itemize}
      \item 7.1 Forward Edge Controls: Discuss future improvements to the
        forward-edge controls in the system.

      \item 7.2 Expanded SBI Implementation: Consider enhancements or additional
        features in the SBI layer.

      \item 7.3 Extended FreeRTOS Implementation: Potential improvements to the
        FreeRTOS integration for better performance or security.
    \end{itemize}

  \item 8. Conclusions - Summarize the key findings of the project, its impact
    on security for RISC-V systems, and areas where the work can be extended or applied
    to other architectures.
\end{itemize}

Maybe from risc-v unpriv memory and another thing Look at priv ISA to see if
there is anything
