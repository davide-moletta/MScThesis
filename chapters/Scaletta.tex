\chapter*{Scaletta}
\label{cha:Scaletta}

\begin{itemize}
  \item Introduction:
    \begin{itemize}
      \item Content: This chapter introduces the background of the project and
        its motivation. It should describe the overall purpose of the project and
        why it is significant in the context of RISC-V development and security.
        You could also briefly discuss the project’s objectives, your approach,
        and the structure of the thesis.

      \item Sections to include:
        \begin{itemize}
          \item Motivation: Explain why RISC-V was chosen, its relevance in the industry,
            and the importance of extending its security features.

          \item Objectives: Outline what you aim to achieve with this project (e.g.,
            developing a secure RISC-V platform).

          \item Thesis Structure: Give a brief outline of how the thesis is
            structured.
        \end{itemize}
    \end{itemize}

  \item RISC-V: this chapter serves to give readers the necessary background on
    RISC-V. It should cover the core components and architectural details,
    setting the foundation for understanding the project specifics later on.
    \begin{itemize}
      \item Content and Sections:

        \begin{itemize}
          \item Architecture Overview: A brief introduction to the RISC-V
            architecture and its design philosophy.

          \item Specifics: Detailed explanation of the base instruction sets and
            the modularity of RISC-V.

          \item Extensions: Discuss the standard and optional extensions (e.g., M,
            A, F, D extensions) and their importance.

          \item Registers \& CSRs: Explain the general-purpose registers and
            Control and Status Registers (CSRs) crucial for RISC-V operation.

          \item Privilege Levels: Detail the different privilege levels (user,
            supervisor, and machine) and their roles in the RISC-V architecture.

          \item Interrupts and Traps: Discuss how the RISC-V architecture
            handles interrupts and traps, including exceptions.

          \item Physical Memory Protection (PMP): Explain the PMP feature and
            how it is used to manage memory access permissions, ensuring security.

          \item Toolchain: Discuss the software toolchain used for development, including
            the compiler, assembler, linker, debugger, and any simulation/emulation
            tools (e.g., GCC, Spike, or QEMU). Explain how these tools interact
            with the RISC-V platform and the project’s codebase.
        \end{itemize}
    \end{itemize}

  \item Project: In this chapter, you introduce your specific project, its goals,
    and its design methodology. You should provide an overview of how your
    implementation extends or modifies the RISC-V architecture to enhance
    security or functionality.
    \begin{itemize}
      \item Content and Sections:

        \begin{itemize}
          \item Formalization: Describe the formal specification of the project
            and its theoretical foundations.

          \item Specifics: Go into the details of your implementation and
            modifications to the RISC-V architecture, highlighting any changes or
            extensions made.

          \item Interrupts Service Routine (ISR): Explain how interrupts are
            managed in your project and any customizations you implemented for
            security or efficiency.

          \item Shadow Stack: Describe the shadow stack mechanism implemented
            for protection against return-oriented programming (ROP) attacks and
            other control-flow integrity issues.

          \item CFG: Describe how the CFG is used to track and validate program
            execution paths. Explain how it helps maintain control-flow integrity
            by monitoring jumps and returns, ensuring they follow predefined
            patterns and preventing malicious redirection of control flow.

          \item PMP (Physical Memory Protection): Detail your PMP implementation
            and how it secures the memory space in your project.

          \item Forward and Backward Controls: Explain the security mechanisms
            you implemented to protect against control-flow attacks (e.g., validating
            jumps, function returns).

          \item Proof of Concept (PoC): Illustrate a practical demonstration or
            simulation of your project showcasing its features and functionality.

          \item Proof of Work (PoW): If relevant, discuss any workload or performance
            measurements you used to validate the efficiency or scalability of your
            implementation.

          \item Instrumentation: Detail the instrumentation techniques used to
            gather runtime information about the system’s behavior. Explain how instrumentation
            was used for debugging, testing, or verifying the correctness and security
            of your implementation.
        \end{itemize}
    \end{itemize}

  \item Security Analysis and Validation: Adding this chapter helps to discuss
    the security aspects of your implementation in more detail, evaluating its effectiveness
    and potential vulnerabilities.
    \begin{itemize}
      \item Content:
        \begin{itemize}
          \item Threat Model: Define the threat model considered during
            development (e.g., hardware attacks, software vulnerabilities).

          \item Testing Methodologies: Describe how you tested the security
            features of your implementation (e.g., using simulation tools or specific
            attack scenarios).

          \item Results and Analysis: Provide a detailed analysis of the
            outcomes from your security tests, including any benchmarks or performance
            metrics.

          \item Limitations: Discuss the limitations of your security mechanisms
            and areas for potential improvement.
        \end{itemize}
    \end{itemize}

  \item Performance Analysis: This chapter assesses the performance of your
    implementation, focusing on time and space efficiency, and evaluating any
    overhead introduced by security mechanisms.

    \begin{itemize}
      \item Content:
        \begin{itemize}
          \item Time Performance: Measure the time overhead of your security
            mechanisms (e.g., interrupt handling latency, shadow stack
            operations) and compare it against a baseline (e.g., a standard RISC-V
            implementation without these features).

          \item Memory Overhead: Discuss the space overhead introduced by the
            security features (e.g., memory used for the shadow stack, PMP
            configuration data).

          \item Optimization Techniques: Outline any optimization methods
            applied to minimize performance costs and discuss their
            effectiveness.

          \item Results and Analysis: Present a detailed analysis of the
            performance data, using graphs, tables, or benchmarks to visualize the
            impact of the implemented features.
        \end{itemize}
    \end{itemize}

  \item FreeRTOS: This chapter covers how you integrated FreeRTOS with your RISC-V
    implementation. FreeRTOS is often used in embedded systems, and integrating
    it can showcase your system’s compatibility and real-world application
    potential.
    \begin{itemize}
      \item Content:
        \begin{itemize}
          \item Porting FreeRTOS: Explain the steps and challenges faced while
            porting FreeRTOS to your customized RISC-V platform.

          \item Tasks and Scheduling: Discuss how the tasks and scheduling are
            handled within FreeRTOS on your system.

          \item Interrupt Management: Detail how FreeRTOS interacts with the ISR
            system you implemented.

          \item Memory Management: Highlight how FreeRTOS uses PMP and other
            security mechanisms in your system.
        \end{itemize}
    \end{itemize}

  \item Future works: This chapter explores potential future improvements and
    extensions for your project, based on the work completed so far.
    \begin{itemize}
      \item Content and Sections:

        \begin{itemize}
          \item Supervisor Binary Interface (SBI) Implementation: Discuss the
            possibility of adding SBI support, explaining how it could enhance functionality
            or performance.

          \item FreeRTOS Implementation Enhancements: Outline additional
            features or optimizations for the FreeRTOS integration, such as enhanced
            scheduling algorithms or memory management improvements.

          \item Additional Security Features: Suggest other security mechanisms
            that could be integrated, such as advanced hardware security features
            or cryptographic accelerators.
        \end{itemize}
    \end{itemize}

  \item Conclusions: The concluding chapter should summarize the main
    achievements of the project, the challenges encountered, and the lessons
    learned.
    \begin{itemize}
      \item Content:
        \begin{itemize}
          \item Summary of Work: Briefly recap the objectives and the main
            contributions of the thesis.

          \item Reflection on Challenges: Reflect on the technical challenges
            you faced and how they were addressed or resolved.

          \item Implications and Impact: Discuss the impact of your work, both in
            the academic field and potentially in industry.

          \item Closing Thoughts: Offer any closing remarks on the significance
            of open-source hardware (like RISC-V) in advancing technology and
            security.
        \end{itemize}
    \end{itemize}
\end{itemize}