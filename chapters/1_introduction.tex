\chapter{Introduction}
\label{cha:introduction}

In recent years the world of embedded devices grew exponentially, they are now integral
part of our everyday life. From home assistants to smart furniture, embedded devices
are used in many fields to accomplish diverse tasks. Moreover, they are often connected
to the internet creating what is nowadays called Internet of Things (IoT). Given
the amount of devices and, in some cases, their importance the interest of
attackers in these small computers grew together with their use. Creating botnets,
exfiltrating sensitive information and disrupting systems are only few of the
possible reasons an attacker could be interested in taking control over an
embedded device. This brings significant security issues that make the security analysis
of embedded devices important.

In this thesis we will see project name, an infrastructure designed to protect
RISC-V based microcontrollers. The focus of this project is to provide essential
security features to any unsafe code running on an embedded device. The main
objective is to enforce Control Flow Integrity (CFI) to ensure that the program
follows the expected paths and that its execution isn't hijacked.

In the following chapter an introduction to the RISC-V Instruction Set
Architecture (ISA) will be discussed, showcasing the main concept needed to understand
the infrastructure underlying the project. After that, the project's implementation
will be shown, highlighting its security features and how they are developed. Moreover,
a Proof of Concept (PoC) will demonstrate how the project addresses threats and
why it works. Furthermore, we will see a threat analysis and a performance
analysis in which the weak points of the project are depicted. In these sections
we will also see why the project effectively poses as a possible solution for the
protection of embedded devices. Lastly, we will see how FreeRTOS could be implemented
into the project to allow more complex programs to run in the secure environment.
