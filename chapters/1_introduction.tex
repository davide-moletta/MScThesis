\chapter{Introduction}
\label{cha:introduction}

This chapter serves as an introduction to \textit{STEERED: Secure Trusted Execution
Environment for RISC-V-based Embedded Devices}, a new security infrastructure
for \textit{RISC-V}-based embedded devices. The proposed solution aims to
provide a Trusted Execution Environment (TEE) to allow the execution of
untrusted user code while guaranteeing that Control Flow Integrity is not
compromised. Moreover, \textit{STEERED} aims at providing an efficient and lightweight
infrastructure that complies with the limited hardware capabilities of embedded devices.

\section{Problem}
\label{sec:intro_context}

Embedded devices are now an integral part of daily life, powering a wide range of
applications that span from smart home assistants and appliances to connected furniture.
These compact computing units have expanded across numerous domains, supporting diverse
tasks that enhance convenience, efficiency, and connectivity in everyday environments.
A defining feature of embedded devices is their network connectivity, which links
them to the broader Internet of Things (IoT) ecosystem. According to the ``\textit{State
of IoT Summer 2024}'' report, $16.6$ billion devices were connected by the end
of $2023$, a figure expected to rise to $18.8$ billion during $2024$\footnote{\url{https://iot-analytics.com/number-connected-iot-devices/}}.
These devices perform critical functions across industries ranging from consumer
electronics to industrial automation and healthcare. However, they are often resource-constrained
and designed to perform specific tasks with minimal power and computational resources,
making them inherently vulnerable to cyber threats. In $2022$, approximately
$112$ million IoT cyberattacks were reported globally\footnote{\url{https://1nce.com/en-us/resources/news/blog/iot-cybersecurity-landscape}},
and this figure continues to escalate, with the first half of $2024$ witnessing a
$107\%$ increase in attacks as highlighted by \textit{SonicWall}'s ``\textit{2024
Mid-Year Cyber Threat Report}''\footnote{\url{https://www.sonicwall.com/threat-report}}.
Consequently, embedded device security is an increasingly pressing concern, as attacks
on these devices can result in data breaches, unauthorized system control, or disruptions
to essential services.

Among the potential threats, we have Software-based, Network-based, and Side-based
attacks that an attacker may use to impact the Confidentiality, Integrity, and
Availability of a system. One notable example of a Network-based attack is the \textit{Man-In-The-Middle}
(MITM) attack. In this scenario, a malicious actor intercepts the communication
between two parties, thereby gaining unauthorized access to sensitive data such as
cryptographic keys, login credentials, and personal information. This breach can
lead to severe consequences, including data theft, identity fraud, and loss of trust
in communication channels. Additionally, when an attacker manages to compromise a
significant number of devices, they can form what is known as a \textit{Botnet}.
This network of infected devices can be orchestrated to launch large-scale
\textit{Denial of Service} (DoS) attacks against external systems. In a DoS
attack, the botnet floods a targeted server or network with excessive traffic,
overwhelming it and rendering it unable to process legitimate requests. This disruption
can lead to service outages, financial losses, and damage to the organization's reputation.

With all this information it is easy to understand why embedded devices have become
a suitable target for cyber attacks and why this is a security concern. It would
be relatively ``easy'' to think of implementing state-of-the-art security
solutions to protect these devices, however, the problem is that many of them
have limited hardware resources as they may be designed to carry out simple
tasks. This means we must apply security measures within the device's capabilities
without affecting its performance. This is a challenge because, in specific cases,
there may be the need to design tailored security features to protect device-specific
components.

\section{Goals}
\label{sec:intro_goals}

The primary objective of the \textit{STEERED} project is to establish a robust
software-based execution environment that facilitates the secure execution of
untrusted user code. This is particularly crucial in scenarios where running potentially
malicious or unverified code can pose significant risks to system integrity and security.
A key focus of the project is to ensure that Control Flow Integrity remains
intact throughout the execution process, thereby preventing any unauthorized modifications
to the flow of control within the program. This protection is vital in
safeguarding against various control-flow tampering attacks that could exploit
vulnerabilities in the code's execution path.

Additionally, considering that the target platform for this project includes embedded
devices, which often have constrained hardware resources, special emphasis has been
placed on employing optimization techniques. These techniques are designed to
minimize the impact of the execution environment on both memory consumption and
execution performance. Our aim is to enable the secure execution of untrusted code
without imposing excessive overhead that could hinder the functionality of resource-limited
embedded systems.

Furthermore, throughout the development process, we have adhered to the open and
modular design principles inspired by \textit{RISC-V}. This commitment to modularity
and openness has resulted in a highly adaptable infrastructure that can be easily
modified to suit a variety of deployment scenarios. By fostering a modifiable
framework, we aim to streamline the deployment process, making it faster and
more efficient in diverse environments, thereby enhancing the overall utility and
effectiveness of the \textit{STEERED} execution environment.

Overall, \textit{STEERED} poses itself as a novel infrastructure which
incorporates Control Flow Integrity to safeguard the control paths within
embedded \textit{RISC-V} systems. This proposal aims to provide a valid
alternative to \textit{ARM}'s \textit{TrustZone}\footnote{\url{https://www.arm.com/technologies/trustzone-for-cortex-m}},
a hardware-based security technology integrated into \textit{ARM Cortex-M} processors,
designed to create a trusted execution environment for handling sensitive
operations and critical data. During this thesis we will highlight the main differences
between \textit{ARM}'s \textit{TrustZone} and \textit{STEERED}, showing how the latter
can effectively contribute to the current state-of-the-art of security mechanisms
to protect embedded devices.

\section{Structure}
\label{sec:intro_structure}

Following this introduction, chapter \ref{cha:riscv} presents the foundational concepts
of the \textit{RISC-V} Instruction Set Architecture (ISA). Such a chapter focuses
only on essential information needed for a deeper understanding of the architectural
basis of the project.

Chapter \ref{cha:project} details the implementation specifics of \textit{STEERED}'s
infrastructure, discussing critical features such as forward and backward edge control
mechanisms and the use of Physical Memory Protection (PMP) to create secure memory
spaces. To validate the effectiveness of \textit{STEERED} in real-world scenarios,
the third chapter also introduces a Proof of Concept (PoC). This PoC will exemplify
how the project addresses critical security threats, illustrating its impact on
device resilience and system integrity.

A subsequent threat analysis, discussed in chapter \ref{cha:ta}, will examine
the types of attacks the project is designed to defend against, assessing its effectiveness
and identifying any potential vulnerabilities that may need further consideration.
In this part of the thesis, we analyze areas of improvement for \textit{STEERED}
while highlighting its strengths.

A performance analysis is also provided in chapter \ref{cha:pa}, where the trade-offs
between security and system efficiency are evaluated. This analysis will help
highlight the strengths and limitations of the project and explain why it
represents a feasible solution for enhancing the security of \textit{RISC-V}-based
embedded devices. During this chapter we showcase test results for time and
memory consumption, showcasing the average overhead of \textit{STEERED}. Lastly,
we provide insights on optimization ideas to further increase the project's
performance.

Additionally, the thesis explores the potential integration of a Real-Time Operating
System (RTOS) for embedded devices into \textit{STEERED} during chapter
\ref{cha:rtos}. In particular, we propose the integration of \textit{FreeRTOS} and
\textit{Zephyr RTOS}, two widely-used RTOSes. This integration could extend the
project's applicability, enabling more complex programs to operate within the secure
environment provided by \textit{STEERED}. By supporting an established RTOS like
\textit{FreeRTOS}, the project could broaden its reach and utility, making it
suitable for a larger variety of use cases in the embedded device domain.

Finally, in chapter \ref{cha:future}, we discuss future enhancements we plan to
implement in \textit{STEERED}. In this chapter, we mainly focus on security and performance-oriented
solutions that could broaden the security of \textit{STEERED} while reducing its
impact on the device's performance.

In summary, this thesis presents a novel approach to securing \textit{RISC-V}
microcontrollers, leveraging Control Flow Integrity to protect against control-flow
attacks and providing a secure foundation for running potentially unsafe code.
Through detailed technical exploration, demonstration, and analysis, this work aims
to contribute a significant advancement to the field of embedded device security,
addressing both current and emerging threats in the IoT landscape.
