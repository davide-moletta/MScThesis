\chapter{Introduction}
\label{cha:introduction}

This chapter serves as an introduction to this thesis, where we present a new security
infrastructure for \textit{RISC-V}-based embedded devices. The proposed solution
aims to provide security features to mitigate control flow hijacking attacks and
allow the execution of untrusted user code. This result is accomplished by
enforcing Control Flow Integrity on the target code. Moreover, great focus has
been put on providing an efficient and lightweight infrastructure that complies
with the limited hardware capabilities of embedded devices.

\section{Problem}
\label{sec:intro_context}

Embedded devices are now an integral part of daily life, powering a wide range
of applications, from smart home assistants and appliances to connected furniture.
These compact computing units have expanded across numerous domains, supporting diverse
tasks that enhance convenience, efficiency, and connectivity in everyday environments.
A defining feature of embedded devices is their network connectivity, which links
them to the broader Internet of Things (IoT) ecosystem. According to the ``\textit{State
of IoT Summer 2024}'' report, $16.6$ billion devices were connected by the end
of $2023$, a figure expected to rise to $18.8$ billion during $2024$\footnote{\url{https://iot-analytics.com/number-connected-iot-devices/}}.
These devices perform critical functions across industries ranging from consumer
electronics to industrial automation and healthcare. However, they are often resource-constrained
and designed to perform specific tasks with minimal power and computational resources,
making them inherently vulnerable to cyber threats. In $2022$, approximately
$112$ million IoT cyberattacks were reported globally\footnote{\url{https://1nce.com/en-us/resources/news/blog/iot-cybersecurity-landscape}},
and this figure continues to escalate, with the first half of $2024$ witnessing a
$107\%$ increase in attacks as highlighted by \textit{SonicWall}'s ``\textit{2024
Mid-Year Cyber Threat Report}''\footnote{\url{https://www.sonicwall.com/threat-report}}.
Consequently, embedded device security is an increasingly pressing concern, as attacks
on these devices can result in data breaches, unauthorized system control, or disruptions
to essential services.

Among the potential threats, we have Software-based, Network-based, and Side-based
attacks that an attacker may use to impact the Confidentiality, Integrity, and
Availability of a system. One notable example of a Network-based attack is the \textit{Man-In-The-Middle}
(MITM) attack. In this scenario, a malicious actor intercepts the communication
between two parties, thereby gaining unauthorized access to sensitive data such as
cryptographic keys, login credentials, and personal information. This breach can
lead to severe consequences, including data theft, identity fraud, and loss of trust
in communication channels. Another example is the threat posed by control flow
hijacking attacks, a family of cyber attacks that aim at disrupting the normal execution
flow of a program to execute arbitrary code or gain control of the target device.
These types of attacks may result in additional threats, for example, if an
attacker manages to gain control of a significant number of devices, they can form
what is known as a \textit{Botnet}. This network of infected devices can be orchestrated
to launch large-scale \textit{Distributed Denial of Service} (DDoS) attacks
against external systems. In a DDoS attack, the botnet floods a targeted server
or network with excessive traffic, overwhelming it and rendering it unable to process
legitimate requests. This disruption can lead to service outages, financial losses,
and damage to the organization's reputation.

With all this information it is easy to understand why embedded devices have
become a suitable target for cyber attacks and why this is a major security
concern. It would be relatively ``easy'' to think of implementing state-of-the-art
security solutions to protect these devices, however, the problem is that many of
them have limited hardware resources as they may be designed to carry out simple
tasks. This means we must apply security measures within the device's
capabilities without affecting performance. This is a challenge because, in
embedded systems security, we often need to design tailored security features to
protect the device while maintaining high efficiency. Here, we discuss how we aim
to contribute to the world of embedded security with the presented project.

\section{Goals}
\label{sec:intro_goals}

This project introduces innovative security features designed specifically for
embedded \textit{RISC-V}-based systems. The primary objective of this project is
to establish a robust software-based Control Flow Integrity enforcer that allows
secure execution of untrusted user code. This is particularly crucial in scenarios
where running potentially malicious or unverified code can pose significant risks
to system integrity and security. A key focus of the project is to ensure that
Control Flow Integrity remains intact throughout the execution process, thereby preventing
any unauthorized modifications to the flow of control within the program. This
protection is achieved by enforcing backward and forward edge protection using
state-of-the-art techniques such as a shadow stack and Control Flow Graph
validation. Moreover, Physical Memory Protection is integrated into the project
to protect critical memory areas.

Additionally, considering that this project aims at protecting embedded devices,
which often have constrained hardware resources, special emphasis has been placed
on employing optimization techniques. These techniques are designed to minimize
the impact of the Control Flow Integrity enforcer on both memory consumption and
execution performance. We aim to enable the secure execution of untrusted code without
imposing excessive overhead that could hinder the functionality of resource-limited
embedded systems.

Furthermore, throughout the development process, we have adhered to the open and
modular design principles inspired by \textit{RISC-V}. This commitment to modularity
and openness has resulted in a highly adaptable infrastructure that can be easily
modified to suit a variety of deployment scenarios. With this, we aim to streamline
the deployment process, making it faster and more efficient in diverse
environments, thereby enhancing the overall utility and effectiveness of provided
security features.

By integrating Control Flow Integrity mechanisms, this project aims to enhance
the security of control paths within \textit{RISC-V} systems. Control Flow
Integrity ensures that the execution of the embedded system remains trusted and
reliable, safeguarding it from potential exploits or vulnerabilities that could compromise
its operations. Through this advancement, the project not only improves the robustness
of the \textit{RISC-V} instruction set architecture but also contributes to the
overall resilience of embedded applications in various critical fields.

\section{Structure}
\label{sec:intro_structure}

Following this introduction, chapter \ref{cha:background} presents a
comprehensive background about the problem, security techniques, and related
works that provide a similar solution to the one presented in this thesis. We start
with the problem's description, severity, and the relevant attacks we aim to mitigate.
Secondly, we discuss state-of-the-art security measures against control flow hijacking
attacks, strongly focusing on Control Flow Integrity. Lastly, we present related
projects discussing what differentiates them from our approach.

In chapter \ref{cha:threatmodel} we provide the threat model and assumptions for
this project. We state the problem and the target of our security measures. Also,
we discuss assumptions we made about the threat scenario and the capabilities of
an attacker.

Chapter \ref{cha:riscv} presents the foundational concepts of the \textit{RISC-V}
Instruction Set Architecture. In this chapter, we focus only on essential
information needed for a complete understanding of the architectural basis of
the project.

Chapter \ref{cha:project} details the implementation specifics of the Control
Flow Integrity enforcer, discussing critical features such as forward and backward
edge control mechanisms and the use of Physical Memory Protection to create secure
memory spaces. To validate the effectiveness of the project in real-world
scenarios, this chapter also introduces a Proof of Concept which will exemplify how
the project addresses critical security threats, illustrating its impact on
device resilience and system integrity.

A subsequent security analysis, discussed in chapter \ref{cha:ta}, will describe
the testing methodologies used to assess the project's effectiveness in
identifying and preventing control flow hijacking attacks. Moreover, we discuss
the security limitations of the Control Flow Integrity enforcer identifying any
potential vulnerabilities that may need further consideration.

A performance analysis is also provided in chapter \ref{cha:pa}, where the trade-offs
between security and system efficiency are evaluated. This analysis will help
highlight the strengths and limitations of the project and explain why it
represents a feasible solution for enhancing the security of \textit{RISC-V}-based
embedded devices. During this chapter we showcase test results for time and
memory consumption, showcasing the average overhead of the project. Lastly, we provide
insights on optimization ideas to further enhance performance.

Additionally, the thesis explores the potential integration of a Real-Time
Operating System (RTOS) for embedded devices during chapter \ref{cha:rtos}. In particular,
we propose the integration of \textit{FreeRTOS} and \textit{Zephyr RTOS}, two widely-used
Real-Time Operating Systems. This integration could extend the project's
applicability, enabling more complex programs to operate under the Control Flow Integrity
enforcer. By supporting an established RTOS like \textit{FreeRTOS}, the project
could broaden its reach and utility, making it suitable for a larger variety of use
cases in the embedded device domain.

Finally, in chapter \ref{cha:future}, we discuss future enhancements we plan to integrate.
In this chapter, we mainly focus on security and performance-oriented solutions that
could broaden the security of the project while reducing its impact on the device's
performance.

In summary, this thesis presents a novel approach to securing \textit{RISC-V} microcontrollers,
leveraging Control Flow Integrity to protect against control flow hijacking
attacks and providing a secure foundation for running potentially unsafe code. Through
detailed technical exploration, demonstration, and analysis, this work aims to
contribute a significant advancement to the field of embedded device security, addressing
both current and emerging threats in the IoT landscape.
