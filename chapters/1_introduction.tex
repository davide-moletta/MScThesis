\chapter{Introduction}
\label{cha:introduction}

This chapter serves as an introduction to \textit{STEERED: Secure Trusted Execution
Environment for RISC-V-based Embedded Devices}, a new security infrastructure
for \textit{RISC-V}-based embedded devices. The proposed solution aims at
providing a Trusted Execution Environment (TEE) to allow execution of untrusted
user code while guaranteeing that Control Flow Integrity is not compromised. Moreover,
\textit{STEERED} aims at providing an efficient and lightweight infrastructure that
complies with the limited hardware capabilities of embedded devices.

\section{Problem}
\label{sec:intro_context}

Embedded devices are now an integral part of daily life, powering a wide range of
applications that span from smart home assistants and appliances to connected furniture.
These compact computing units have expanded across numerous domains, supporting diverse
tasks that enhance convenience, efficiency, and connectivity in everyday environments.
A defining feature of embedded devices is their network connectivity, which links
them to the broader Internet of Things (IoT) ecosystem. According to the ``\textit{State
of IoT Summer 2024}'' report, $16.6$ billion devices were connected by the end
of $2023$, a figure expected to rise to $18.8$ billion during $2024$\footnote{\url{https://iot-analytics.com/number-connected-iot-devices/}}.
These devices perform critical functions across industries ranging from consumer
electronics to industrial automation and healthcare but they are often resource-constrained
and designed to perform specific tasks with minimal power and computational resources,
making them inherently vulnerable to cyber threats. In $2022$, approximately
$112$ million IoT cyberattacks were reported globally\footnote{\url{https://1nce.com/en-us/resources/news/blog/iot-cybersecurity-landscape}},
and this figure continues to escalate, with the first half of $2024$ witnessing a
$107\%$ increase in attacks as highlighted by \textit{SonicWall}'s ``\textit{2024
Mid-Year Cyber Threat Report}''\footnote{\url{https://www.sonicwall.com/threat-report}}.
Consequently, embedded device security is an increasingly pressing concern, as attacks
on these devices can result in data breaches, unauthorized system control, or disruptions
to essential services.

Among the potential threats, we have Software-based, Network-based, and Side-based
attacks that an attacker may use to impact Confidentiality, Integrity, and
Availability of a system. For example, with a \textit{Man-In-The-Middle} (MITM) attack,
an actor could gain access to cryptographic keys and steal sensitive information.
Moreover, if an attacker manages to take control of many devices, creating what is
known as a \textit{Botnet}, he could use such devices to perpetrate \textit{Denial
of Service} (DoS) attacks to external systems.

With all these information it is easy to understand why embedded devices have
become a suitable target for cyber attacks and why this is a security concern. It
would be relatively ``easy'' to think of implementing state-of-the-art security solution
to protect these devices, however, the problem is that many of them have limited
hardware resources as they may be designed to carry out simple tasks. This means
that we must apply security measures that are within the device's capabilities
without affecting its performance. This truly is a challenge because, in
specific cases, there may be the need to design tailored security features to protect
device-specific components.

\section{Goals}
\label{sec:intro_goals}

In this thesis, we introduce \textit{STEERED} (Secure Trusted Execution Environment
for \textit{RISC-V}-based Embedded Devices), an which incorporates Control Flow
Integrity (CFI) to safeguard the control paths within embedded \textit{RISC-V} systems.
This mechanism utilizes a lightweight CFI enforcer, specifically designed to be compatible
with the limited resources of RISC-V microcontrollers. By integrating structures
such as a Shadow Stack and Control Flow Graph (CFG), \textit{STEERED} ensures
that only authorized execution flows occur, thereby blocking potential control-flow
hijacking attempts without compromising device performance.

Following this introduction, Chapter $2$ presents the foundational concepts of
the \textit{RISC-V} Instruction Set Architecture (ISA), essential for
understanding the architectural basis of the project. Chapter $3$ details the implementation
specifics of \textit{STEERED}'s infrastructure, discussing critical features such
as forward and backward edge control mechanisms and the use of Physical Memory Protection
(PMP) to create secure memory spaces. To validate the effectiveness of \textit{STEERED}
in real-world scenarios, Chapter $3$ also introduces a Proof of Concept (PoC). This
PoC will exemplify how the project addresses critical security threats,
illustrating its impact on device resilience and system integrity. A subsequent threat
analysis will examine the types of attacks the project is designed to defend against,
assessing its effectiveness and identifying any potential vulnerabilities that
may need further consideration. A performance analysis is also provided, where
the trade-offs between security and system efficiency are evaluated. This analysis
will help highlight the strengths and limitations of the project and explain why
it represents a feasible solution for enhancing the security of embedded devices.

Finally, the thesis explores the potential integration of \textit{FreeRTOS}, a
widely-used Real-Time Operating System (RTOS) for embedded devices, into \textit{STEERED}.
This integration could extend the project's applicability, enabling more complex
programs to operate within the secure environment provided by the infrastructure.
By supporting an established OS like \textit{FreeRTOS}, the project could broaden
its reach and utility, making it suitable for a larger variety of use cases in
the embedded device domain.

In summary, this thesis presents a novel approach to securing \textit{RISC-V}
microcontrollers, leveraging Control Flow Integrity to protect against control-flow
attacks and providing a secure foundation for running potentially unsafe code.
Through detailed technical exploration, demonstration, and analysis, this work aims
to contribute a significant advancement to the field of embedded device security,
addressing both current and emerging threats in the IoT landscape.