\chapter{Introduction}
\label{cha:introduction}

In recent years, the world of embedded devices has experienced exponential growth,
with these small computing units becoming a fundamental part of daily life. From
smart home assistants and appliances to integrated furniture, embedded devices
are now widely deployed across various domains, performing a wide array of tasks
and transforming everyday environments. A defining feature of these devices is
their connectivity, as many of them are linked to the Internet, creating what is
known as the Internet of Things (IoT). The ``\textit{State of IoT Summer 2024}''
report showed that, by the end of $2023$, there were $16.6$ billion connected
devices and this number is expected to grow to $18.8$ billion during $2024$\footnote{\url{https://iot-analytics.com/number-connected-iot-devices/}}.

While the Internet of Things amplifies the usefulness and capabilities of embedded
devices, it also introduces potential vulnerabilities as they can easily be
reached by outside threats. Given their prevalence and, in some cases, their critical
functions, embedded devices have become attractive targets for attackers who
exploit their accessibility and computational limitations. Threat actors may seek
to compromise these devices to create botnets, exfiltrate sensitive data, or disrupt
systems, among other malicious purposes. In $2022$ there were approximately $112$
million IoT cyberattacks\footnote{\url{https://1nce.com/en-us/resources/news/blog/iot-cybersecurity-landscape}}
while, in the first half of $2024$, we saw a $107\%$ increase in attacks as
stated by the ``\textit{2024 Mid-Year Cyber Threat Report}''\footnote{\url{https://www.sonicwall.com/threat-report}}
published from \textit{SonicWall}. These numbers underscore the urgent need for
robust security measures, making the study and implementation of security protocols
for embedded devices a pressing research area.

However, the problem is that many embedded devices have limited hardware
resources as they may be designed to carry out simple tasks. This means that we must
apply security measures that are within the device's capabilities without affecting
its performance. This is a challenge because, in specific cases, we may need to
design tailored security features to protect devices.

In this thesis, we introduce \textit{project name}, an infrastructure aimed at securing
RISC-V-based microcontrollers. The primary goal of this project is to establish
essential security features that can safeguard any potentially unsafe code
running on an embedded device. Central to this security solution is the enforcement
of Control Flow Integrity (CFI), which ensures that the program's execution
follows its intended path and prevents unauthorized control flow alterations. By
implementing CFI, \textit{project name} can detect and mitigate control-flow hijacking
attacks, enhancing the overall security of the system.

The structure of this thesis is organized to facilitate a comprehensive
understanding of both the theoretical and practical aspects of the project. The following
chapter begins with an introduction to the RISC-V Instruction Set Architecture (ISA),
presenting the foundational concepts necessary for understanding the
architectural structure of \textit{project name}. A detailed overview of the
project's implementation then follows, where specific security features and mechanisms
designed to counteract common attack vectors are discussed. This section will
include explanations of the steps taken to fortify RISC-V microcontrollers
against unauthorized control-flow tampering.

Furthermore, a Proof of Concept (PoC) is included to demonstrate the practical application
and effectiveness of \textit{project name} in real-world scenarios. This PoC will
exemplify how the project addresses critical security threats, illustrating its
impact on device resilience and system integrity. A subsequent threat analysis will
examine the types of attacks the project is designed to defend against,
assessing its effectiveness and identifying any potential vulnerabilities that
may need further consideration. A performance analysis is also provided, where
the trade-offs between security and system efficiency are evaluated. This analysis
will help highlight the strengths and limitations of the project and explain why
it represents a feasible solution for enhancing the security of embedded devices.

Finally, the thesis explores the potential integration of \textit{FreeRTOS}, a
widely-used real-time operating system for embedded devices, into \textit{project
name}. This integration could extend the project's applicability, enabling more
complex programs to operate within the secure environment provided by the
infrastructure. By supporting an established OS like \textit{FreeRTOS}, the project
could broaden its reach and utility, making it suitable for a larger variety of
use cases in the embedded device domain.

In summary, this thesis presents a novel approach to securing RISC-V microcontrollers,
leveraging Control Flow Integrity to protect against control-flow attacks and
providing a secure foundation for running potentially unsafe code. Through detailed
technical exploration, demonstration, and analysis, this work aims to contribute
a significant advancement to the field of embedded device security, addressing both
current and emerging threats in the IoT landscape.