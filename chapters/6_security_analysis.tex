\chapter{Security Analysis}
\label{cha:ta}

This chapter delves into a comprehensive analysis of the various security
vulnerabilities that the project seeks to address. We will thoroughly examine the
methodologies employed to assess the project's resilience against potential threats,
highlighting the rigorous testing and evaluation processes involved. Furthermore,
we will outline the system's limitations, offering a clear depiction of its practical
capabilities and constraints. Finally, we will present recommendations for enhancing
the project's security features, illustrating paths for future improvement and
adaptation to evolving security challenges.

\section{Testing Methodologies}
\label{sec:ta_methodologies}

The evaluation of the project's security has been conducted with both static and
dynamic analysis techniques, tailored to assess the system's resilience to the
threats identified in the threat model described in chapter
\ref{cha:threatmodel}. Moreover, we provide additional analysis specifically
designed to establish the correctness of the Control Flow Graph, shadow stack,
and Physical Memory Protection. The testing procedure has been conducted as follows:

\begin{itemize}
  \item Static Analysis: in the initial phase of our project, we conducted a thorough
    analysis of the source code to uncover any potential flaws or implementation
    errors. This meticulous review allowed us to address all identified static
    issues during our testing process, ultimately leading to a more robust and accurate
    codebase. Furthermore, we carefully outlined all expected scenarios that could
    arise during execution. This preparation laid the groundwork for the
    implementation phase, during which we manually examined each of the standard
    paths to ensure their correctness and reliability;

  \item Control Flow Graph Validation: in this phase, we extracted the Control Flow
    Graph from the target application. Our primary objective was to ensure that
    the extracted graph accurately reflects the legitimate execution paths
    predetermined as acceptable for the application's intended functionality. To
    achieve this, we compared the extracted CFG against these predefined
    legitimate paths. Any discrepancies or deviations from what was expected during
    our testing phase were meticulously flagged as potential vulnerabilities.
    This assessment allowed us to identify areas of concern that could be
    exploited or lead to unexpected behaviors. As part of this rigorous validation
    process, we developed a reliable Control Flow Graph extraction function. This
    function was specifically designed to ensure that the CFG provided a precise
    and accurate representation of the application's control flow. The
    reliability of this extraction is crucial, as it serves as the foundation for
    our subsequent analyses and security assessments, facilitating a deeper
    understanding of the application's behavior and potential security risks;

  \item Shadow Stack Validation: in this step, we meticulously examined the functions
    involved in interacting with the shadow stack mechanism. Our validation
    process included a thorough review of the address storage methods to ensure
    that each address was recorded accurately and securely. We also implemented rigorous
    checks on the popping and matching logic to confirm that it operated correctly,
    thereby guaranteeing the integrity of the return instruction data. This comprehensive
    evaluation aimed to provide a high level of assurance regarding the legality
    of return instructions, preventing potential vulnerabilities in the return
    flow of execution;

  \item Physical Memory Protection Validation: we conducted some tests to check
    the implemented Physical Memory Protection configuration. Such tests involved
    trying to access secure parts of the memory from a low-privilege environment.
    As a result, we determined that each access in read, write, and execute mode
    resulted in the generation of a trap that was then handled by the interrupt vector
    table accordingly. This testing validated that Physical Memory Protection
    effectively restricts access, protecting the integrity of the shadow stack and
    Control Flow Graph from unauthorized modifications. Also, we assessed that higher
    privilege code can effectively access the data structures without generating
    traps. We provide an example of Physical Memory Protection validation in
    Listing \ref{lst:pmpcheck}. Here, we try to call both the \textit{push} and \textit{pop}
    functions of the shadow stack from within the user-mode world. However,
    thanks to the devised PMP, both instructions have been blocked and a trap has
    been generated and handled by the interrupt vector table;

    \begin{lstlisting}[style=CStyle, caption = Physical Memory Protection testing, label={lst:pmpcheck}]
#include "shadow_stack.h"
void user_function() {
 u_int malicious_address = 10;
 push(malicious_address);
 printf("Pushed: %d\n", pop());
}
 \end{lstlisting}

  \item Dynamic Analysis: we performed dynamic testing of edge controls to test
    their secureness. The system was subjected to simulations using a variety of
    inputs designed to leverage control flow hijacking vulnerabilities, such as
    crafted payloads that mimic Return-Oriented Programming and Jump-Oriented
    Programming attacks. With such instructions, we tried to perform
    unauthorized control transfers and, in all cases, the Control Flow Integrity
    enforcer was able to detect the attack and prevent the control transfer. Moreover,
    logging mechanisms were implemented during testing to capture runtime
    behavior, enabling the detection of any deviation from the expected control flow.
    We provide two examples of dynamic testing in Listings \ref{lst:returncheck}
    and \ref{lst:jumpcheck}.

    \begin{lstlisting}[style=CStyle, caption = Return-Oriented Programming simulation attack, label={lst:returncheck}]
int user_function() {
  // Other code ...
  asm("la t0, malicious_func"); // Load addr of malicious code
  asm("sw t0, 12(sp)");         // Substitute return address
  return 0;
}
       \end{lstlisting}

    In the first example, we add the instruction \textit{la t0, malicious\_func}
    to load the address of some malicious code into register \textit{t0}. Then,
    we overwrite the return address of the current function with the instruction
    \textit{sw t0, 12(sp)}\footnote{Note that \textit{12(sp)} is an example as the
    return address is usually stored as this by the compiler.}. Now the current function
    will return to the malicious code as soon as it reaches the return instruction,
    however, thanks to the Control Flow Integrity enforcer we are sure that this
    will not happen as such address was never pushed into the shadow stack.

    \begin{lstlisting}[style=CStyle, caption = Jump-Oriented Programming simulation attack, label={lst:jumpcheck}]
int user_function() {
  // Other code ...
  asm("la t0, malicious_func"); // Load addr of malicious code
  asm("mv a5, t0");             // Substitute jump address
  // JALR instruction
  // Other code ...
  return 0;
}
    \end{lstlisting}

    In the second example instead, we try to tamper with the jump address of an indirect
    jump instruction. By adding the instruction \textit{la t0, malicious\_func} we
    load the address of some malicious code into register \textit{t0}. Then, we
    use the instruction \textit{mv a5, t0} to substitute the destination address
    of the jump instruction\footnote{Note that this is an example as the
    compiler usually uses register \textit{a5} to store the address of indirect jump
    instructions by calling convention.} to perform an unauthorized control
    transfer. However, we know that such a source-destination pair is not part
    of the Control Flow Graph, thus the forward edge control will fail and the instruction
    will not be executed. These simple examples show how the project is able to
    detect and block any type of unwanted control transfer instruction;

  \item Edge Case Validation: In our final phase of testing, we focused on edge case
    validation. This involved a series of comprehensive tests designed to assess
    the system's behavior in unusual or extreme scenarios. Specifically, we
    conducted experiments to observe how the system reacts when the shadow stack
    is either completely empty or full. These conditions were intentionally engineered
    to push the limits of the system, allowing us to identify potential
    vulnerabilities and ensure robust performance under such peculiar
    circumstances. Through these tests, we aimed to gain a deeper understanding
    of the system's stability and reliability when faced with atypical
    operational situations.
\end{itemize}

The project has undergone rigorous testing across a range of potential attack
scenarios and unforeseen circumstances. The results of these tests indicate that
the system demonstrates a robust ability to handle these challenges effectively.
This reliability ensures that the Control Flow Integrity of the code remains
intact, instilling confidence that the system can be trusted to function securely
in various threat environments.

\section{Limitations}
\label{sec:ta_limitations}

While we introduce robust protections against control flow hijacking attacks, certain
limitations must be acknowledged.

Firstly, the project lacks coverage against non-control-flow hijacking attacks. The
system does not mitigate side-channel attacks, such as timing or power analysis,
which require separate countermeasures. Data-oriented attacks that do not rely
on altering the control flow are also outside the scope of this implementation. While
this does not represent a problem for the project development itself as these threats
were out of scope during the definition part of the project, as described in the
threat model, they are indeed a real-life problem. This means that the project
must be paired with other security measures to effectively protect a device from
various types of attacks.

Moreover, we assume software integrity to be guaranteed. The system assumes that
the source code of the project, including the compilers, is trusted and free of
pre-existing vulnerabilities. Compromised source code could undermine the integrity
of the Control Flow Integrity enforcer protections. Furthermore, if the source code
gets compromised we can't ensure its correctness in protecting the Control Flow Integrity
of the code. Thus, additional security measures must be put in place to avoid possible
tampering with the code. Again, as already explained in the threat model, this
is out of scope as we assume that the code is inherently safe.

Additionally, there could be edge cases that have not been discovered which could
lead to potential exploitations of the Control Flow Integrity enforcer. However,
these can be discovered through exhaustive testing and will be part of future development.

Lastly, we can't know if vulnerabilities related to the \textit{RISC-V} Instruction
Set Architecture could lead to the exploitation of the user code. In such a case,
we can't be sure that the project will be able to protect the Control Flow Integrity
and prevent exploitation. For example, the recently discovered hardware vulnerability
in \textit{T-Head's XuanTie C910} and \textit{C920} CPUs named \textit{GhostWrite}\cite{riscvuzz}
(Fabian Thomas, Lorenz Hetterich, et al., 2024) allows an attacker with limited privileges
to read and write from and to physical memory. This type of vulnerability could
not be prevented with the provided security features as a threat actor may be
able to overwrite Physical Memory Protection configurations or the shadow stack
and Control Flow Graph themselves to modify these data structure, leading to a compromisation
of the Control Flow Integrity enforcer.

In summary, we provide robust protection against unauthorized control transfers
in \textit{RISC-V} microcontrollers, particularly effective in embedded
environments where code integrity is paramount. While the model is limited in handling
recursion-heavy applications and relies on Physical Memory Protection for data protection,
it nonetheless establishes a secure foundation for Control Flow Integrity
enforcement, with the potential for further optimization in future iterations.
Moreover, other security mechanisms could be implemented to completely secure an
embedded device.
