\chapter{Control Flow Integrity Enforcer}
\label{cha:project}

This chapter is focused on showcasing this project's development and
implementation. We start by formalizing the project's goals, showing its
purposes and how it could address real-world problems. Section \ref{sec:project_instrumentation}
provides a detailed description of the instrumentation process, describing how
it affects the produced binary and why it has been designed to work in this way.
Section \ref{sec:project_isr} describes how interrupts, exceptions, and edge controls
are handled. Sections \ref{sec:project_ss} and \ref{sec:project_cfg} show how
the shadow stack and Control Flow Graph are implemented, respectively. Moreover,
Section \ref{sec:project_pmp} showcases the configuration for the Physical Memory
Protection used to secure such data structures. In section
\ref{sec:project_controls}, we describe how forward and backward edge controls are
enforced, giving technical insights into their design. Lastly, in Section
\ref{sec:project_poc}, a Proof of Concept is discussed to prove the functioning and
security capabilities provided by this project.

\section{Project Formalization}
\label{sec:project_formalization}

With this project, we aim to provide a secure infrastructure for embedded devices
based on the \textit{RISC-V} Instruction Set Architecture. The main goal is to protect
the device from control flow hijacking attacks such as \textit{Return-Oriented
Programming} or \textit{Jump-Oriented Programming} attacks. These cyber-threats
attempt to alter the normal execution flow of a program by jumping or returning
to unexpected addresses with the intent of executing arbitrary code or
exfiltrating information. This is even more dangerous if we consider that small microcontrollers
often use only the highest privilege mode when executing code, as a result, an attacker
has a high chance of gaining complete control over the target device.

To prevent the described attacks, we enforce Control Flow Integrity on the
device by providing a shadow stack and Control Flow Graph validation. With this security
technique, we ensure that the software follows the expected path, preventing any
unauthorized attempt to alter the execution flow. Moreover, the project provides
instrumenting capabilities to automatize the implementation of any code, making
the use of the infrastructure easy and fast.

Another goal of the provided implementation is to be as lightweight as possible
to meet the performance requirements of less powerful devices. Thus, great
importance is given to the optimization of both space and time consumption.

Lastly, the open and modular \textit{RISC-V} principles have been followed
during the design phase, leading to a highly customizable project. This ensures
that the project can be modified and adapted to work in any circumstance.

To sum up, the presented project offers an easy and lightweight infrastructure that
makes it possible to run untrusted code in a secure environment, protecting the
execution path of the software and ensuring that any control flow tampering
attempt will be detected and blocked by the Control Flow Integrity enforcer.

\section{Project Specifics}
\label{sec:project_specifics}

In this section, we outline the details of the project, including the tools and files
used. This project is published on \textit{GitHub}\cite{repo} under the \textit{GNU
GPL-3.0} license\cite{gpl3}, making it freely available to anyone.

The project's basic configuration for bare-metal flashing is derived from Sergey
Lyubka's project \textit{mdk}\cite{mdk}, however, we propose a custom flashing solution
specifically developed for this project. The only requirements to run and flash
the code are \textit{Python v3+} and a toolchain to cross-compile code. During
development, we utilized the \textit{riscv-none-elf}\cite{toolchain} toolchain but,
with the proper adjustments, any toolchain can be used. Moreover, since the development
and testing has been carried out on \textit{Espressif's ESP32-C3-DevKitM-1}\cite{esp32c3},
we need to include the \textit{esputil}\footnote{Already available in the
\textit{GitHub} repository as a submodule\cite{esputil}.} executable to interact
with the board\footnote{Note that if we wish to change the target board, all vendor-specific
files must replace \textit{Espressif}'s files.}.

\begin{wrapfigure}
  [17]{l}{.25\textwidth}
  \centering
  \def\stackalignment{l}\stackunder{ \includegraphics[width=\linewidth]{images/workingtree.png} } %
  {\scriptsize \parbox[t]{\linewidth}{Source: \href{https://github.com/davide-moletta/RISC-V-TE}{GitHub repository}}}
  \caption{Working Tree}
  \label{fig:workingtree}
\end{wrapfigure}

Figure \ref{fig:workingtree} depicts the project's working tree, where:

\begin{itemize}
  \item \textit{esp32} contains the boot configuration and linker script for
    general \textit{Espressif}'s boards\footnote{Provided by Sergey Lyubka};

  \item \textit{esp32c3} contains the boot configuration and linker script for
    \textit{Espressif's ESP32-C3}\footnote{Provided by Sergey Lyubka but
    severely modified during development.};

  \item \textit{esputil} contains \textit{Espressif}'s utils used to manage the board;

  \item \textit{src/cfi/usercode} contains the source files of the untrusted
    code. The files inside this folder will be the target for code instrumentation
    and will be used to produce the secure binary;

  \item \textit{src/cfi} contains the source files for the shadow stack, Control
    Flow Graph, interrupt vector table, and machine setup, which will be
    carefully described later;

  \item \textit{toolsExtra} contains the \textit{Python} scripts \textit{instrumenter.py},
    \textit{flasher.py}, and \textit{CFGextractor.py} used to instrument, flash,
    and extract the Control Flow Graph respectively (detailed description in Section
    \ref{sec:project_instrumentation}).
\end{itemize}

Inside \textit{src/cfi}, we find the code responsible for managing the machine
mode operations. File \textit{main.c} is responsible for enabling interrupts, managing
privilege modes, and setting up both the Physical Memory Protection (detailed description
in Section \ref{sec:project_pmp}) and the interrupt vector table (detailed description
in Section \ref{sec:project_isr}). File \textit{intr\_vector\_table.c} is
responsible for managing interrupts and exceptions as well as performing both
forward and backward edge controls (detailed description in Section \ref{sec:project_controls}).
Files \textit{cfg.c} and \textit{shadow\_stack.c} hold the Control Flow Graph and
the shadow stack configurations, respectively (detailed description in Sections \ref{sec:project_cfg}
and \ref{sec:project_ss}). Lastly, \textit{ij\_logger.c} is used to retrieve indirect
jump addresses thanks to a logger.

\subsection{User-Code Importing}
\label{subsec:project_ucodeimport}

During development, we put great importance on making the importing of user code
easy. Such a process requires only two steps, firstly, we need to copy the
needed files inside \textit{src/cfi/usercode}. Secondly, we just need to modify the
pre-uploaded file \textit{user\_entry.c} located inside the same folder. The purpose
of this file is to make this process faster by providing a predefined function to
modify. By looking at Listing \ref{lst:userentry} we can see that we just need
to call the first function we want to execute of the imported code and include
the related file. \\ \begin{lstlisting}[style=CStyle, caption = \textit{user\_entry.c} file, label={lst:userentry}]
#include "user_file.h"

void user_mode_entry_point() {
  printf("\n\n--- Start of user code ---\n\n");
  start_u_code(); // Call first user function here
  printf("\n\n--- End of user code ---\n\n");
}
\end{lstlisting}

\section{Code Instrumentation}
\label{sec:project_instrumentation}

Code instrumentation is the process of modifying software (usually binary or
assembly code) by inserting instructions to perform specific tasks such as performance
analysis. Instrumentation plays a critical role in this project as it allows
modification of the untrusted code in a simple yet effective way. Moreover, this
automatizes the process, leading to faster development and reduced number of errors.
The whole instrumentation procedure is depicted in Figure
\ref{fig:instrumentation}. \\
\begin{figure}[htbp]
  \centering
  \includegraphics[width=.9\linewidth]{images/instrumentation.png}
  \caption{Flow of the code instrumentation procedure}
  \label{fig:instrumentation}
\end{figure}
\\

In this section, we explain how the \textit{Python} scripts work and how they
can be used to instrument, build, and run the code.

The script \textit{flasher.py} can be run with the command \textit{python3
flasher.py command}, where \textit{command} can be:
\begin{itemize}[noitemsep]
  \item \textit{build}: used to build the source files and produce the binary.
    This command does not provide security features;

  \item \textit{run}: used to build and run the binary on the target device.
    This command does not provide security features;

  \item \textit{clear}: used to remove \textit{.bin}, \textit{.elf}, \textit{.s},
    and \textit{.log} files in the directory;

  \item \textit{secure-build}: used to instrument and build the source files and
    produce the secure binary with the security features;

  \item \textit{secure-run}: used to instrument, build, and run the secure binary
    on the target device with the security features.
\end{itemize}

As we can see, the instrumentation only happens with \textit{secure-build} and
\textit{secure-run} commands. Normal building and running commands have been
implemented to make a comparison between untrusted and trusted code.

The following subsections explain in detail all the steps that occur during the
instrumentation phase.

\subsection{Instrumentation for Logging}
\label{subsec:logging}

Firstly, \textit{flasher.py} takes all the source files and compiles them into assembly
files with the command \textit{riscv-none-elf-gcc -S CFLAGS\footnote{Utilized
flags are: \textit{-W -Wall -Wextra -Werror -Wundef -Wshadow -pedantic -Wdouble-promotion
-ffixed-a7 -fno-common -Wconversion -march=rv32imc\_zicsr -mabi=ilp32 -O1 -ffunction-sections
-fdata-sections -fno-builtin-printf}.} source files}. After that, assembly files
of the untrusted user code are passed to \textit{instrumenter.py} for instrumentation.

In the first step, the code is instrumented with logging capabilities to
retrieve indirect jump destinations. This is done by searching for indirect jump
instructions (\textit{JALR}) with the regex \textit{$\backslash$b(jalr)$\backslash$b$\backslash$s+($\backslash$w+)}
which retrieves any occurrence of \textit{jalr register}\footnote{Usually the
compiler uses register \textit{a5} for \textit{JALR} instructions.} and allows us
to retrieve the register used to perform the jump.

Once we retrieve the source register for the \textit{JALR} instruction, we add the
block of code depicted in Listing \ref{lst:loggingblock} before the jump.

\begin{wrapfigure}
  [33]{r}{0.25\textwidth}
  \setlength{\intextsep}{0pt}
  \begin{minipage}{0.25\textwidth}
    \begin{lstlisting}[style=Assembly, caption = Logging code block, label={lst:loggingblock}]
addi sp, sp, -40
sw a5, 4(sp)
sw a4, 8(sp)
sw a2, 12(sp)
sw a1, 16(sp)
sw a0, 20(sp)
sw s0, 24(sp)
sw s1, 28(sp)
sw s2, 32(sp)
sw s3, 36(sp)
mv a1, src_reg
auipc a0, 0
addi a0, a0, 38
call print_reg
lw a5, 4(sp)
lw a4, 8(sp)
lw a2, 12(sp)
lw a1, 16(sp)
lw a0, 20(sp)
lw s0, 24(sp)
lw s1, 28(sp)
lw s2, 32(sp)
lw s3, 36(sp)
addi sp, sp, 40
jalr src_reg
 \end{lstlisting}
  \end{minipage}
\end{wrapfigure}

This effectively allows us to save the state of the machine and call the
function \textit{print\_reg} passing the destination address and the program counter
of the \textit{JALR} instruction as arguments. The destination address is simply
retrieved from the source register of the jump instruction and is saved inside register
\textit{a1}. On the other hand, the source address of the jump instruction is
computed by loading the current program counter with \textit{auipc a0, 0} and by
adding $38$\footnote{Note that $38$ is the distance in Bytes from the
instruction that loads the \textit{pc} to the \textit{JALR} instruction.} to it.

When called, the function \textit{print\_reg} prints a string like \textit{Source:
src\_address - Destination: dst\_address} where \textit{src\_address} and
\textit{dst\_address} are the source and destination addresses of the \textit{JALR}
instruction, respectively.

\subsection{Control Flow Graph Extraction}
\label{subsec:project_cfgextraction}

After the instrumentation for logging is completed, the assembly files are assembled
and linked to produce the binary with the command \textit{riscv-none-elf-gcc
LINKFLAGS\footnote{Utilized flags are: \textit{-Tesp32c3/link.ld -nostdlib -nostartfiles
-Wl,--gc-sections}} assembly files -o output.bin}.

Moreover, if during the instrumentation, indirect jumps were found in the code, we
perform a ``simulation''\footnote{The simulation consists of running the code on
the target device transparently.} of the execution to retrieve the logging of
the \textit{print\_reg} function.

In any case, the next step involves the extraction of the Control Flow Graph of
the code by calling \textit{CFGextractor.py}. The extraction is performed in two
phases:
\begin{itemize}
  \item Dynamic phase: in the dynamic phase, we parse the output retrieved from the
    simulation to create source-destination pairs of addresses. Note that in
    this phase, addresses are also adjusted by removing the size of the logging block
    from their value. This is done because, during the simulation, we injected
    the logging block before each jump, thus increasing the size of the binary.
    As a result, the retrieved pairs would not match the actual addresses of the
    final binary. So, we compare each address with each block and perform the comparison
    $\textit{address}> \textit{block}$. If the address is higher than the
    current block, this means that the block appeared before the instruction so,
    we add $1$ to the number of blocks. Once we perform each comparison, we can
    compute
    $\textit{final address}= \textit{retrieved value}- (\textit{block size}\footnote
    {Note that the size of each logging block is $56$ Bytes.}* \textit{number of
    blocks})$;

  \item Static phase: in the static phase, we simply parse the dump file produced
    with the command \textit{riscv-none-elf-objdump -D output.bin} searching for
    \textit{JAL} instructions. These direct jumps are statically defined in the
    dump file with the following format \textit{source address: jal destination address}
    so we just need to extract the needed values. Each time a \textit{JAL} instruction
    is found, the pair source-destination is directly added to the CFG.
\end{itemize}

Once the Control Flow Graph is extracted, all the pairs are ordered in ascending
order, firstly by source and then by destination. This is done because, with
indirect jump instructions, we could have more destinations that share the same source.
This process is performed to simplify and optimize the search at run time (detailed
description in Section \ref{sec:project_cfg}). After this step, the execution is
returned to \textit{instrumenter.py} for the final instrumentation phase.

\subsection{Instrumentation for Forward and Backward Edge Controls}
\label{subsec:project_instrcontrols}

In this last instrumentation phase, we need to add code blocks that allow us to perform
forward and backward edge controls. Such blocks must be added before every
direct jump, indirect jump, and return instruction. To do so, we parse the
assembly files and search for the target instructions thanks to the Regular
Expression (\textit{regex}) functions depicted in Table \ref{tab:regexes}.

\begin{table}
  \centering
  \begin{tabular}{|c|c|}
    \hline
    \textbf{Regex}                                                                      & \textbf{Use}                            \\
    \hhline{==} \textit{$\backslash$b(call)$\backslash$b$\backslash$s+($\backslash$w+)} & Used to find \textit{JAL} instructions  \\
    \hline
    \textit{$\backslash$b(jalr)$\backslash$b$\backslash$s+($\backslash$w+)}             & Used to find \textit{JALR} instructions \\
    \hline
    \textit{$\backslash$b(jr)$\backslash$b$\backslash$s+($\backslash$w+)}               & Used to find \textit{RET} instructions  \\
    \hline
  \end{tabular}
  \caption{\textit{Regex} functions used to find target instructions}
  \label{tab:regexes}
\end{table}

Depending on the instruction we find during parsing, we do the following:
\begin{itemize}
  \item \textit{JAL} instructions: if we find a direct jump instruction, we need
    to add the code depicted in Listing \ref{lst:dirjumpblock} before the target.
    This code loads the address of the target function in register \textit{a7}
    and then performs an \textit{ECALL} instruction (detailed functioning
    explained in \ref{sec:project_isr}). We use the \textit{load address} (\textit{la})
    instruction since we need to load the address to which the retrieved label
    refers;

  \item \textit{JALR} instructions: if we find an indirect jump instruction, we
    need to add the code depicted in Listing \ref{lst:indirjumpblock} before the
    target. This code copies the address stored in the target register into register
    \textit{a7} and then performs an \textit{ECALL} instruction. We use the
    \textit{move} (\textit{mv}) instruction since we need to copy the address
    from one register to another register;

  \item \textit{RET} instructions: if we find a return instruction, we need to
    add the code depicted in Listing \ref{lst:retblock} before the target. This code
    copies the return address stored in the return address register into register
    \textit{a7} and then performs an \textit{ECALL} instruction. We use the
    \textit{add immediate} (\textit{addi}) instruction since we need to copy the
    address from one register to another and add $1$. In Section \ref{sec:project_isr}
    we will see why, in this case, we need to add $1$ to the return address.
\end{itemize}

\begin{lstlisting}[style=Assembly, caption = Direct jump code block, label={lst:dirjumpblock}]
la a7, {target_function}
ecall
jal {target_function}
\end{lstlisting}

\begin{lstlisting}[style=Assembly, caption = Indirect jump code block, label={lst:indirjumpblock}]
mv a7, {target_register}
ecall
jalr {target_register}
\end{lstlisting}

\begin{lstlisting}[style=Assembly, caption = Return code block, label={lst:retblock}]
addi a7, {return_address_register}, 1
ecall
addi {return_address_register}, a7, -1
ret {return_address_register}
\end{lstlisting}

As soon as the script finishes parsing the dump file, it injects the previously crafted
Control Flow Graph into the \textit{cfg.c} file.

After this second instrumentation ends, the modified assembly files are assembled
and linked by \textit{flasher.py} to produce the secure binary file. Lastly, if
we used the \textit{secure-run} command, the binary file is flashed on the
target device for execution.

\section{Machine Setup}
\label{sec:project_setup}

Setting up the machine correctly is fundamental to ensure the correctness of the
provided security features. Here, we describe each step needed to configure the machine
properly. \\
\begin{lstlisting}[style=CStyle, caption = Machine setup, label={lst:setup}]
int main(void) {
  asm("la t0, interrupt_vector_table"); // Load vector table address
  asm("ori t0, t0, 1");                 // Set MODE bit to 1
  asm("csrw mtvec, t0");                // Load the address in MTVEC

  asm("csrr t0, mstatus");  // Load MSTATUS in t0
  asm("li t1, 0xFFFFE7FF"); // Load user mode status in t1
  asm("and t0, t0, t1");    // Change MPP bits to user mode
  asm("or t0, t0, 8");      // Change MIE bits to 1
  asm("csrw mstatus, t0");  // Write new MSTATUS

  asm("la t0, user_mode_entry_point"); // Load user mode entry point
  asm("csrw mepc, t0");                // Write new MEPC

  // PMP configuration
  asm("la t0, interrupt_vector_table"); // Load end of first region
  asm("srli t0, t0, 2");                // Shift right
  asm("csrw pmpaddr0, t0");             // Load address in CSR

  asm("la t0, shadow_stack");           // Load end of second region
  asm("srli t0, t0, 2");                // Shift right
  asm("csrw pmpaddr1, t0");             // Load address in CSR

  asm("la t0, .machine_setup");         // Load end of third region
  asm("srli t0, t0, 2");                // Shift right
  asm("csrw pmpaddr2, t0");             // Load address in CSR

  asm("li t0, 0x90000000");             // Load end of fourth region
  asm("srli t0, t0, 2");                // Shift right
  asm("csrw pmpaddr3, t0");             // Load address in CSR

  asm("li t0, 0x0F0B0F0B");             // Load configuration mask
  asm("csrw pmpcfg0, t0");              // Write conf to CSR

  asm("mret"); // Jump to user code in U mode
}
\end{lstlisting}

Referring to Listing \ref{lst:setup}, which depicts a simplified version of the
\textit{main} function, we see that in lines \textit{2-4}, we load the address
of the interrupt vector table in a temporary register. Then, we set \textit{MODE}
bit to $1$ to enable vectored mode, and we load the proper value inside \textit{mtvec}.

In lines \textit{6-10}, we load the value of \textit{mstatus} inside a temporary
register before modifying it. Firstly, we use a bitmask to modify \textit{machine
previous privilege} bits to U-mode. After that, we set \textit{machine interrupt
enable} bits to enable interrupts. Lastly, we load the new \textit{mstatus}.

In lines \textit{12-13}, we load the address of the user mode entry point\footnote{The
first function that needs to be executed in user mode.} inside \textit{mepc}.

After that, the Physical Memory Protection is configured (Detailed description in
Section \ref{sec:project_pmp}), and the instruction \textit{mret} is used to
return execution to the address stored in \textit{mepc} which, in this case, is the
first function of the user code.

Overall, this machine configuration allows us to correctly manage traps thanks
to the interrupt vector table and to configure a secure PMP as well as correctly
returning execution to the user code.

\section{Trap Management}
\label{sec:project_isr}

The purpose of this section is to showcase how the interrupt vector table is implemented
and how interrupts and exceptions are handled within the project. Most
importantly, we will see how forward and backward edge controls are enforced.

The interrupt vector table is defined inside \textit{intr\_vector\_table.c}. As already
explained its address is loaded in \textit{main.c} and stored inside \textit{mtvec}
(\ref{subsec:riscv_mtvec}) with \textit{MODE} set to vectored. This means that every
asynchronous interrupt will set the program counter to the base address of the interrupt
vector table plus $4$ times the cause of the interrupt. Any other interrupt and
exception is trapped inside the function \textit{synchronous\_exception\_handler}
which redirects to the correct function depending on the trap cause. Listing
\ref{lst:intrtable} depicts the actual implementation of the interrupt vector
table.

Most of the exceptions and interrupts are not currently handled, and when invoked,
they just log a message describing which trap was taken before resuming the
execution. This design choice has been made for two reasons. The first is that
such implementation would be out of scope since we aim to provide the bare
minimum implementation for enforcing Control Flow Integrity. The second instead is
that we do not need those handlers, and we leave space for implementation-specific
requirements. For example, if a future implementation makes use of external
interrupts, the developer would just need to insert the desired handling in the correct
function inside the \textit{intr\_vector\_table.c} file. As explained, this
implementation provides the required security features while leaving space for
customization.

Since \textit{ECALL}s are defined as exceptions in \textit{RISC-V}, the
interrupt vector table is responsible for managing them. For this reason, the
only implemented function inside \textit{intr\_vector\_table.c} is the handler for
U-mode \textit{ECALL}s. This implementation is tailored to manage forward and
backward edge controls. However, since \textit{ECALL}s can be used for different
purposes, the handler is designed to be expandable. Table \ref{tab:ecalls}
depicts the currently allowed ecalls.
\begin{table}
  \centering
  \begin{tabular}{|c|c|c|}
    \hline
    \textbf{Code}                & \textbf{Use}          & \textbf{Description}                  \\
    \hhline{===} $1$             & Reserved              & Used to terminate execution           \\
    \hline
    \textit{destination address} & Forward edge control  & Used to check the destination address \\
    \hline
    $\textit{return address}+ 1$ & Backward edge control & Used to check the return address      \\
    \hline
  \end{tabular}
  \caption{Encoding of currently allowed \textit{ECALL} values}
  \label{tab:ecalls}
\end{table}
\\ \begin{lstlisting}[style=CStyle, caption = Interrput Vector Table, label={lst:intrtable}]
void interrupt_vector_table(void) {
  asm volatile("j synchronous_exception_handler");
  asm volatile("j isr_supervisor_software");
  asm volatile("j isr_reserved");
  asm volatile("j isr_machine_software");
  asm volatile("j isr_user_timer");
  asm volatile("j isr_supervisor_timer");
  asm volatile("j isr_reserved");
  asm volatile("j isr_machine_timer");
  asm volatile("j isr_user_external");
  asm volatile("j isr_supervisor_external");
  asm volatile("j isr_reserved");
  asm volatile("j isr_machine_external");
  asm volatile("j isr_reserved");
}
\end{lstlisting}

Currently, the U-mode \textit{ECALL} handler is implemented as depicted in
Listing \ref{lst:ecallhandler}. As it is possible to see, we encoded the \textit{ECALL}
code and the address required for edge controls into a single value. This design
has been used to minimize the use of registers. Another solution would have been
to use a register, say \textit{a7}, to hold the \textit{ECALL} code, and another
register, say \textit{a6}, to hold the address to check. However, since legal addresses
must be even we can encode in one register both the value of the \textit{ECALL}
code and the address using the otherwise unused least significant bit (Figure \ref{fig:ecall}
shows the encoding of register \textit{a7}). When the interrupt vector table
needs to check which code is used, we can just see if the value in \textit{a7}\footnote{In
Listing \ref{lst:ecallhandler} register \textit{a7} is represented by the parameter
\textit{ecode}.} is even or odd and, depending on the result, we can decide which
operation to perform\footnote{Even values will lead to a forward edge check
while odd values will lead to a backward edge check.}. When we need to check for
a return address, we first subtract $1$ from it to obtain the original value of
the address, and then we perform the control. \\
\begin{figure}[htbp]
  \centering
  \includegraphics[width=.9\linewidth]{images/ecall_code.png}
  \caption{Encoding of register a7 during \textit{ECALL}}
  \label{fig:ecall}
\end{figure}
\\

Note that the use of register \textit{a7} to hold \textit{ECALL} codes is purely
a design choice and, to ensure that such register is not tampered with by the
compiler, it has been ``disabled'', meaning that the compiler will never use
such register. This choice allows us to be sure that only the interrupt vector table
and the code blocks that we inject modify the value of \textit{a7}. Another solution
would have been to let the compiler use register \textit{a7} and, during the instrumentation
phase, check for its usage. When used, we could inject more instructions to
store the previous value in the stack and then retrieve it after the edge
control. However, since we wanted to inject as few instructions as possible to
avoid impacting too much on the performance, the first solution has been implemented.
\\
\begin{lstlisting}[style=CStyle, caption = U-mode \textit{ECALL} handler, label={lst:ecallhandler}]
void esr_handler_U_mode_ecall(u_int ecode, u_int mepc) {
  if (ecode == 1) {
    code_termination();
  } else if ((ecode % 2) == 0) {  // Forward edge control
    u_int source = mepc + 4;      // Source address

    if (check(source, ecode)) {   // If address is inside the CFG
      if (push(source + 2) != 1){ // Try push into shadow stack
        code_termination();       // Push failed
      }
    } else {
      code_termination();         // CFG check failed
    }
 } else if ((ecode % 2) != 0) {  // Backward edge control
    u_int pop_addr = pop();       // Popped address

    // Compare popped address with return address
    if (pop_addr == 0 || pop_addr != ecode - 1) {
      code_termination();         // Check failed
    }
 } else {
  // Undefined ecode, log and return
 }
}
\end{lstlisting}

If one wishes to add a new custom \textit{ECALL} code for other purposes, they just
need to put a control before the handler checks if the \textit{ecode} is even or
not. Say that we want to add code $2$ to perform a specific operation. To do so,
we just need to add a check like \textit{else if(ecode == 2) \{perform operation\}}
after the first check, if the check fails, the code will eventually check if the
code is even or not and perform the corresponding edge control.

It is easy to see how this implementation effectively allows the management of
any trap while addressing the problem and leaving space for possible future
customization of the interrupt vector table.

Forward edge controls are performed thanks to the Control Flow Graph and are
further explained in section \ref{sec:project_cfg} while backward edge controls
are performed thanks to the shadow stack and are further explained in section
\ref{sec:project_ss}.

\section{Shadow Stack}
\label{sec:project_ss}

The shadow stack holds the values of return addresses, and it is used to check
the correctness of each \textit{RET} instruction. File \textit{shadow\_stack.c} holds
the configuration for the shadow stack. The development of said data structure
took inspiration from the formally verified idea proposed in the article ``\textit{Work
in progress: A formally verified shadow stack for RISC-V}'' (Matthieu Baty, Guillaume
Hiet et al., $2022$)\cite{shadowstack}. In their article, the writers demonstrate
a methodology for formally specifying and verifying the correctness of a shadow stack
implemented in \textit{RISC-V}. Their approach involves the use of formal methods
and the hardware description language \textit{Kôika}\cite{koika} to prove the integrity
of the shadow stack against return address manipulation. The authors also
discuss the isolation of the shadow stack from regular memory access, ensuring robust
protection against tampering, and provide a verified framework that halts
execution if a violation is detected, such as a stack overflow or an invalid return
address. This rigorous approach highlights the importance of formal verification
in designing secure hardware systems, which served as a foundational inspiration
for our implementation.

The shadow stack is implemented as a standard Last-In-First-Out (\textit{LIFO})
stack. This design choice has been made to provide a data structure with fast lookups\footnote{Lookups
in the shadow stack are performed in $\mathcal{O}(1)$.} while maintaining the
ability to use limited space when return addresses are not needed. This implementation
is possible only because the last jump instruction that is performed in a code will
always be the first to return. Otherwise, it would be necessary to store each return
address and its relative jump instruction to know which address we need to check
each time. This allowed us to build an effective and fast data structure that
consumes a variable amount of memory, rarely affecting the execution performance.

The shadow stack is statically defined as a \textit{C} struct that contains an
array that can hold up to $63$ addresses and a pointer to the top index of the
stack (Listing \ref{lst:ss} shows the described struct). Moreover, the file
\textit{shadow\_stack.c} allows two operations, \textit{push} and \textit{pop}
where:
\begin{itemize}
  \item \textit{Push} is used during a forward edge control. If such control succeeds,
    the return address related to the jump instruction is pushed into the shadow
    stack and stored\footnote{Addresses are stored in the shadow stack as
    \textit{unsigned integers}.} for later use. In this case, we just use the stack
    pointer to determine if the shadow stack can accommodate more addresses. If
    the shadow stack is full and no other addresses can be stored, we terminate execution
    immediately as we can't provide the necessary data for the next backward edge
    control. Otherwise, the address is added to the top of the stack, and the
    execution continues normally;

  \item \textit{Pop} is used when we need to perform a backward edge control. When
    this happens, an address is removed from the top of the shadow stack. Afterward,
    we match the removed address with the user-provided return address to decide
    whether the return instruction is considered legal or not\footnote{Note that
    when we use \textit{pop} we remove the element from the stack since we won't
    need it in the future and, by doing this, we can free some space in the data
    structure.}. Note that before removing the address, we check if the shadow stack
    is empty, and, if that is the case, we terminate execution immediately since
    the code is trying to perform an unexpected return instruction.
\end{itemize}

The shadow stack provides no other functions for two main reasons. Firstly, with
\textit{push} and \textit{pop}, we are effectively able to make the controls we need.
Secondly, by adding extra functions, not only do we increase the binary size, but
we could also accidentally insert weak code inside the binary that may lead to unexpected
behavior.

Note that the size of the shadow stack can be adjusted to fit every need by simply
modifying the value of \textit{MAX\_SIZE} inside \textit{shadow\_stack.c}. For
example, if we know that our code will never nest for more than $5$ times, we
can reduce the size of the shadow stack to $6$ since we are sure that we will
never have more than $5$ jump instructions consecutively. \\ \begin{lstlisting}[style=CStyle, caption = shadow stack definition inside \textit{shadow\_stack.c}, label={lst:ss}]
#define MAX_SIZE 63

typedef struct {
  u_int addresses[MAX_SIZE]; // Array used to store return addresses
  int top;                   // Pointer to the top of the stack
} SStack;
\end{lstlisting}

\section{Control Flow Graph (CFG)}
\label{sec:project_cfg}

The Control Flow Graph holds the pair source-destination for each direct and indirect
jump of the code and it is used to check the correctness of each \textit{JAL}
and \textit{JALR} instruction. As already said, the CFG of the binary is extracted
during the instrumentation phase and injected into the file \textit{cfg.c} which
holds the configuration for the Control Flow Graph (Listing \ref{lst:cfgdefinition}
depicts the current representation of the Control Flow Graph). \\ \begin{lstlisting}[style=CStyle, caption= Definition of the Control Flow Graph inside \textit{cfg.c}, label={lst:cfgdefinition}]
u_int cfg[][2] = {{src, dst}, ...}; // Source-destination pairs
size_t cfg_size = 1;                // Size of the CFG
\end{lstlisting}

It is worth noting that the CFG is related only to control transfer instructions
of the untrusted code. This is because we need to enforce Control Flow Integrity
only on U-mode code, thus not saving the CFG of all the binary helps in reducing
memory usage. Moreover, the Control Flow Graph does not hold the values of
return addresses, this is because return instructions are checked thanks to the shadow
stack, thus we do not need to save those values in the CFG. These design choices
allowed us to create a tailored structure that is both fast and efficient in space
usage.

Inside \textit{cfg.c}, we can see the structure of the Control Flow Graph, which
is composed of a two-dimensional array where each index represents an edge. Such
an edge is stored in a small array that contains the source address in the first
position and the destination address in the second position. So, the CFG holds a
pair \textit{source-destination} at each position, moreover, as explained in
section \ref{sec:project_instrumentation}, the pairs are ordered in non-decreasing
order during the instrumentation phase. The \textit{cfg.c} file provides only the
\textit{check} function, which asks for a pair of addresses as input and determines
whether such a pair is part of the Control Flow Graph or not. \textit{check} is implemented
using a binary search algorithm with a custom comparison function (Listing
\ref{lst:binsearch} depicts a representation of the compare function). The
custom compare function was created because we needed to perform the binary
search with two values (source and destination addresses), thus a normal comparison
would not work in this scenario. With this function, we first search for the source
address, and then, if we find a match, we search for the destination address. Note
that this approach is possible only because the list was previously ordered,
otherwise, the binary search would not work correctly and we would need to perform
a linear search, which would result in a time complexity of $\mathcal{O}(n)$. Lastly,
if the binary search finds a match, the \textit{check} function returns a
positive value, and the jump instruction is considered legal.

Since the CFG will not change during execution, we can define it statically and
provide no functions to add or remove elements. Again, the reason behind this
choice is to reduce memory usage and to avoid the implementation of unused
functions. \\ \begin{lstlisting}[style=CStyle, caption= Comparison function for binary search, label={lst:binsearch}]
int compare(const int* A, const int* B) {
  for (int i = 0; i < 2; ++i) {
    if (A[i] < B[i]) return -1;
    if (A[i] > B[i]) return 1;
  }
  return 0;
}
\end{lstlisting}

This implementation of the Control Flow Graph effectively reduces space
consumption to the bare minimum while providing fast lookups with a time
complexity of $\mathcal{O}(\log{n})$. Another solution could be to use a Hash Table
to store the address pairs. In this case, we would reduce the time required to access
the CFG to $\mathcal{O}(1)$ for any size of the CFG since accesses to Hash Tables
are performed in constant time. However, this works only with big enough Hash Tables,
otherwise, we could face many collisions, and the time required to find the
correct entry would increase. This means that we would need to allocate a lot of
space to store a few addresses, leading to a waste of memory usage since most of
the entries would be empty. Still, this alternative solution could be perfect
for situations in which we care more about reducing time overhead rather than
space overhead.

\section{Memory Layout}
\label{sec:layout}

The purpose of this section is to discuss the memory layout of the binary (Listing
\ref{lst:linker} depicts a simplified version of the linker script used to
compile the code). \\
\begin{lstlisting}[style=CStyle, caption= Simplified linker script, label={lst:linker}]
MEMORY {
  iache (rwx): ORIGIN = 0x4037c000, LENGTH = 16k
  iram  (rwx): ORIGIN = 0x40380400, LENGTH = 32k
  dram  (rw): ORIGIN = 0x3fc80000 + LENGTH(iram), LENGTH = 128k
}

ENTRY(_start)

SECTIONS {
  .interrupt_vector_table: {*(.interrupt_vector_table)} > iram

  .intr_service_routines: {*(.intr_service_routines)} > iram

  .shadow_stack: {*(.shadow_stack)} > iram

  .cfg: {*(.cfg)} > iram

  .machine_setup: {*(.machine_setup)} > iram

  .text: {*(.text) *(.text*)} > iram

  .ij_logger: {*(.ij_logger)} > iram

  .data: {*(.data*) *(.sdata*) *(.srodata*) *(.rodata*)} > dram
}
\end{lstlisting}

Each part of the codebase has been assigned to separate sections as follows:
\begin{itemize}
  \item Section \textit{.interrupt\_vector\_table} holds the interrupt vector
    table function described in \ref{lst:intrtable};

  \item Section \textit{.intr\_service\_routines} holds the exception and
    interrupt handlers as well as the functions to interact with the shadow
    stack and Control Flow Graph;

  \item Section \textit{.shadow\_stack} holds the data structure related to the
    shadow stack;

  \item Section \textit{.cfg} holds the data structure related to the Control
    Flow Graph;

  \item Section \textit{.machine\_setup} holds the boot setup functions and the
    \textit{main} function;

  \item Section \textit{.text} holds the untrusted code;

  \item Section \textit{.ij\_logger} holds the function \textit{logger} used to log
    indirect jump addresses during the simulation.
\end{itemize}

This configuration has been implemented to increase granularity between all the parts
of the code. Moreover, as we will see in section \ref{sec:project_pmp}, such
configuration allows for an easier and more effective definition of Physical
Memory Protection.

Execution starts at function \textit{\_start}, which is located inside the file
\textit{boot.c}. Such function initializes the device and transfers execution to
the \textit{main.c} file, which sets up all the M-mode configurations before
starting the user code.

\section{Physical Memory Protection Configuration}
\label{sec:project_pmp}

In this project, the role of Physical Memory Protection is to protect the shadow
stack and the Control Flow Graph from unauthorized access and modification (the PMP
definition is depicted in Listing \ref{lst:setup}). To ensure that both these data
structures are secured, four memory regions have been created (each region's configuration
can be seen in Table \ref{tab:pmpregions}).

The four regions are designed to cover different memory sections and provide
different privileges:
\begin{itemize}
  \item Region $1$: the first region covers all the memory space between
    $0x00000000$ and the end of the \textit{.data} region. This part of memory
    has been granted with read and write privileges, since this region contains only
    static data there is no need to allow execution. To configure this region, we
    used the instruction \textit{la t0, interrupt\_vector\_table} to load the
    address of the \textit{.interrupt\_vector\_table} section which follows the \textit{.data}
    section. Afterward, we used \textit{srli t0, t0, 2} to make a right shift as
    it is required by \textit{RISC-V}. Lastly, we load the address in \textit{pmpaddr0}
    with the instruction \textit{csrw pmpaddr0, t0}. Note that we load the address
    of the section that follows \textit{.data} because with a TOR configuration
    the upper address determines the end of the PMP section;

  \item Region $2$: the second region covers the interrupt vector table and all
    the exceptions and interrupts handlers as well as CFG and shadow stack
    functions, since this region is accessed only in Machine mode and we need to
    execute instructions, we need to provide read, write, and instruction
    execution privileges. To configure this region, we used the instruction
    \textit{la t0, shadow\_stack} to load the address of the \textit{.shadow\_stack}
    section, which follows the \textit{.interrupt\_service\_routine} section. Afterward,
    we used \textit{srli t0, t0, 2} to make a right shift as it is required by \textit{RISC-V}.
    Lastly, we load the address in \textit{pmpaddr1} with the instruction \textit{csrw
    pmpaddr1, t0};

  \item Region $3$: the third region is the one that covers both the shadow
    stack and the Control Flow Graph. For this reason, we need to restrict
    privileges to read and write only. We can't set this region as read-only since
    we need to modify the shadow stack during execution. Note that the \textit{.shadow\_stack}
    and \textit{.cfg} sections only cover the data structures related to the
    shadow stack and Control Flow Graph, respectively. \textit{Push}, \textit{pop},
    and \textit{check} are categorized as ``handlers'', thus those functions are
    inserted in the \textit{.intr\_service\_routine} section. This is because we
    need instruction execution privileges for said functions to work. To configure
    this region, we used the instruction \textit{la t0, .machine\_setup} to load
    the address of the \textit{.machine\_setup} section which follows the
    \textit{.cfg} section. Afterward, we used \textit{srli t0, t0, 2} to make a
    right shift as it is required by \textit{RISC-V}. Lastly, we load the address
    in \textit{pmpaddr2} with the instruction \textit{csrw pmpaddr2, t0};

  \item Region $4$: the fourth and last region covers all the addresses from the
    start of the main function up to the end of the memory. This region also includes
    user code and, since we need to execute the code, we need to configure this region
    with read, write, and instruction execution privileges. To configure this
    region, we used the instruction \textit{li t0, 0x900000000} to load the
    value $0x90000000$, which represents the end of the memory. Afterward, we used
    \textit{srli t0, t0, 2} to make a right shift as it is required by \textit{RISC-V}.
    Lastly, we load the address in \textit{pmpaddr3} with the instruction
    \textit{csrw pmpaddr3, t0}. Note that there is no need to separate user code
    from the machine setup as they need the same privileges during execution\footnote{Referring
    to PMP privileges, not actual execution privileges.}.
\end{itemize}

Lastly, we use the instructions \textit{li t0, 0x0F0B0F0B} and \textit{csrw
pmpcfg0, t0} to load the configuration of the four memory regions. The value
$0x0 B0F0B0F$ represents the privileges and type of each memory region. Referring
to section \ref{sec:riscv_pmp}, we have:
\begin{itemize}
  \item $0F$ or $00001111$: used to configure the fourth region as a TOR section
    with read, write, and instruction execution privileges;

  \item $0B$ or $00001011$: used to configure the third region as a TOR section
    with read and write privileges;

  \item $0F$ or $00001111$: used to configure the second region as a TOR section
    with read, write, and instruction execution privileges;

  \item $0B$ or $00001011$: used to configure the first region as a TOR section
    with read and write privileges.
\end{itemize}

Note that the value of the configuration is read backward because \textit{RISC-V}
interprets this value as little endian.

This Physical Memory Protection configuration effectively helps to prevent unauthorized
modifications to critical components such as the shadow stack and the Control Flow
Graph. Thanks to this, we are sure that whatever value we read from these data structures
will be safe and trustable. Note that any access made from U-mode to the third
memory region will result in either an \textit{Instruction Access Fault} or an \textit{Illegal
Instruction} exception.

\begin{table}
  \centering
  \begin{tabular}{|c|c|c|c|c|}
    \hline
    \textbf{Region}    & \textbf{Region Start}                             & \textbf{Region End}              & \textbf{Type} & \textbf{Privileges} \\
    \hhline{=====} $1$ & $0x00000000$                                      & \textit{.data} ($0x40380400$)    & TOR           & R-W                 \\
    \hline
    $2$                & \textit{.interrupt\_vector\_table} ($0x40380400$) & \textit{.intr\_service\_routine} & TOR           & R-W-X               \\
    \hline
    $3$                & \textit{.shadow\_stack}                           & \textit{.cfg}                    & TOR           & R-W                 \\
    \hline
    $4$                & \textit{.machine\_setup}                          & $0x90000000$                     & TOR           & R-W-X               \\
    \hline
  \end{tabular}
  \caption{PMP memory regions}
  \label{tab:pmpregions}
\end{table}

\section{Forward and Backward Edge Controls}
\label{sec:project_controls}

The most critical job carried out by the infrastructure is validating jump and
return instructions with the U-mode \textit{ECALL} handler located inside the \textit{intr\_vector\_table.c}
file. Given the delicacy of Control Flow Integrity and its importance for this project,
great attention has been given to the correct implementation of forward and
backward edge controls.

As already explained in sections \ref{sec:project_ss} and \ref{sec:project_cfg},
forward and backward edge controls are performed thanks to the Control Flow Graph
and shadow stack, respectively. However, in this section, each control will be carefully
explained to show how it works and why it provides a certain degree of security to
the project.

\subsection{Forward Edge Controls}
\label{subsec:forward}

A forward edge control is performed to check whether a direct or indirect jump
instruction is trying to transfer control from a legal address to a legal
address. In our case, legality is determined by the fact that the jump stays in
the address range related to the user code and that the pair source destination
is contained in the Control Flow Graph.

To perform a forward edge control, we need three things. Firstly, we need the source
address of the jump instruction. Secondly, we need the destination addresses of the
jump instruction. Thirdly, we need a trusted oracle that tells us if the pair source-destination
is legal. As already explained, the destination address is retrieved from \textit{a7},
which is passed as the \textit{ECALL} code. The source address instead can be retrieved
from \textit{mepc} \ref{subsec:mepc}. As we have seen, when a trap is taken, \textit{mepc}
is written with the address of the instruction that was interrupted. Since in \textit{RISC-V}
an \textit{ECALL} generates a trap, we can just add $4$ to the address stored in
\textit{mepc} to retrieve the source address of the jump instruction\footnote{We
add $4$ since it is the size of an \textit{ECALL} instruction in \textit{RISC-V}.}.
Now that we have the needed pair of addresses to check we can use the Control Flow
Graph as the oracle. As already explained, the CFG is computed before compilation
and is securely stored in a safe memory region thanks to Physical Memory Protection.
This means that any attempt of unauthorized modification is instantly blocked, thus
we can trust the data provided by the CFG and use it as an oracle.

When we need to perform a forward edge control, we just send the source and
destination addresses to the \textit{check} function of the CFG. Such a function
performs a binary search inside the Control Flow Graph to see whether that
specific pair is part of the original CFG or not. If the search succeeds, the
function returns a positive value and the jump is considered legal while, if the
search does not succeed, the instruction is aborted and the execution terminates
to prevent any possible damage.

Whenever a forward edge control succeeds, we know that the related jump instruction
will eventually return so, we need to store its return address inside the shadow
stack. To do so, we need to retrieve the return address but, since a jump will always
return to its next instruction, we can just compute $\textit{source address}+ 2$
to retrieve it\footnote{We add $2$ since it is the size of compressed \textit{JAL}
and \textit{JALR} instructions in \textit{RISC-V}.}. After that, the return address
is pushed into the shadow stack, and the execution is resumed with the
interrupted jump instruction. Note that the execution is resumed if and only if the
shadow stack has room for the pushed address, otherwise we abort the operation
and terminate execution.

\subsection{Backward Edge Controls}
\label{subsec:backward}

A backward edge control is performed to check whether a return instruction is trying
to transfer control from a legal address to a legal address. In our case, legality
is determined by the fact that the return stays in the address range related to the
user code and that the return address is the last address that was pushed into the
shadow stack.

To perform a backward edge control we need two things. Firstly, we need the address
to which the return instruction is trying to transfer control. Secondly, we need
a trusted oracle that tells us if such an address is legal. As we have seen, the
destination address is retrieved from \textit{a7}, which is passed as the \textit{ECALL}
code. However, we must remove $1$ from the address retrieved from \textit{a7} since,
in the least significant bit, we store the \textit{ECALL} code. Similar to the Control
Flow Graph, the shadow stack is secured in a protected memory space and is
unaffected by unauthorized modification attempts. Even in this case, we can
trust the shadow stack and use it as an oracle.

When we need to perform a backward edge control, we just need to \textit{pop} an
address from the shadow stack and match the destination address of the return
instruction with the popped one. If the addresses are different, the execution
is terminated as the code is trying to return to an unauthorized address, while if
the addresses are equal, we can consider the return instruction legal and resume
execution with the interrupted return instruction.

Whenever a backward edge control succeeds, we must do nothing since the value has
already been removed from the shadow stack and no other modifications are needed.

We must note that backward edge controls could be implemented in another way. If
we need to control a return address and there is a mismatch, we could force the user
code to return to the address stored in the shadow stack, which we know is safe.
While this solution enforces Control Flow Integrity securely, we can't be sure
that the stack\footnote{Referring to the normal stack used to store data and not
the shadow stack.} has not been compromised and, thus terminating execution is a
much safer choice.

\section{Proof of Concept}
\label{sec:project_poc}

In this section, we provide a Proof of Concept to showcase how the discussed
project effectively provides security capabilities to a non-trusted and insecure
code. Figure \ref{fig:functioning} depicts an abstraction of the project's flow where
green and red boxes depict trusted and untrusted components, respectively. Moreover,
green arrows are used to represent a successful edge control, while red arrows represent
an unsuccessful one. \\
\begin{figure}[htbp]
  \centering
  \includegraphics[width=.9\linewidth]{images/functioning.png}
  \caption{Execution Flow Abstraction}
  \label{fig:functioning}
\end{figure}
\\

We have already seen that the \textit{Boot and Startup} section is device-specific
and it is used to configure the hardware. \textit{Machine code init} instead, is
used to configure machine registers and then load the Physical Memory Protection
configuration. This part of the code is considered trusted and we can be sure
the machine will be configured properly and no security measures are needed. Note
that this is true from the infrastructure's perspective as these sections could be
tampered with by modifying the source code of \textit{boot.c} and \textit{main.c}
files. However, this is out of the project's scope and would require additional security
measures like human control prior to the building procedure.

The same reasoning is true for the \textit{CFI Enforcer} section, which is responsible
for managing edge controls as well as other interrupts and exceptions. Even in
this case, the source code has to be modified to tamper with the security features.

On the other hand, the \textit{User code} section is untrusted and we can't make
any assumptions on its functioning. The code could be well-written and somewhat
secure but, it could even be full of flaws, and we must prevent any control flow
hijacking attack that could be perpetrated through it. To do so, we use forward
and backward edge controls together with the shadow stack and the Control Flow
Graph.

As already said, the \textit{shadow stack} and the \textit{Control Flow Graph} sections
are the most critical components of the project. We must secure them since they
serve as oracles and we must be able to trust the data they contain to perform
forward and backward edge controls correctly. Since the CFG is configured statically,
we are sure that it can't be modified during execution, the shadow stack instead
is designed to change since we need to \textit{push} and \textit{pop} values from
it. However, since we protected the shadow stack with Physical Memory Protection,
we can be sure that only privileged code has access to it and any other access
generates a trap that is handled through the interrupt vector table. This means that,
even if one tries to add or remove arbitrary values from the stack, the operation
will be aborted immediately and our trust in the shadow stack remains intact.

Below we list and discuss every possible scenario that could happen during execution:
\begin{itemize}
  \item Forward edge control: as soon as a forward edge control is requested, we
    check that the pair source-destination is valid thanks to the \textit{check}
    function of the Control Flow Graph. In this situation, we could receive:
    \begin{itemize}
      \item Legal pair: in this case, the user code is trying to perform a normal
        jump instruction and we know that the added instructions will always send
        to the interrupt vector table a legal pair. As a consequence, the jump will
        be considered safe and the return address will be pushed into the shadow
        stack. Note that since we compute the return address each time instead
        of trusting the one provided by the user code, we are sure that the value
        inserted in the stack is correct and we can trust it;

      \item Non-legal pair: suppose now that an attacker is trying to perform an
        unauthorized jump instruction, in this case, the Control Flow Graph will
        either contain the provided pair of addresses or not. If the pair is contained
        in the CFG, the jump instruction is allowed, however, this means that
        the attacker is trying to jump from a valid source address to a valid
        destination address, meaning that the operation is secure and can be performed.
        On the other hand, if the attacker is trying to perform an unauthorized jump
        instruction, we are sure that the Control Flow Graph will not contain
        the pair source-destination provided by the attacker since we are sure
        that the CFG can't be modified without triggering a trap. In this case,
        the unauthorized jump is detected and the execution terminates
        immediately. The only way in which an attacker would be able to perform an
        unauthorized jump is by injecting the ``infected'' pair of addresses in the
        Control Flow Graph before the configuration of the Physical Memory Protection
        but, this requires the modification of the source code and, again, this
        is out of the project's scope.
    \end{itemize}

  \item Backward edge control: whenever a backward edge control is requested, we
    check that the return address provided by the user code is the one we are expecting
    by popping the last value that was inserted in the shadow stack. In this
    situation, we could receive:
    \begin{itemize}
      \item Legal return address: in this case, the user code is trying to perform
        a normal return instruction and we know that the added instructions will
        always send to the interrupt vector table a legal return address. As a
        consequence, the return instruction will be considered safe, and the
        execution is resumed with the interrupted return;

      \item Non-legal return address: again, let's say that an attacker is trying
        to perform an unauthorized return instruction. In this case, the provided
        return address and the popped one will either match or not. If the
        addresses are the same, the operation is allowed, however, this means that
        the attacker is trying to return to a valid destination address, meaning
        that the operation is secure and can be performed. On the other hand, if
        the provided address is not legal, it will for sure differ from the one we
        pop from the shadow stack for two reasons. Firstly, we trust the addresses
        we push into the shadow stack as they are computed each time to guarantee
        correctness. Secondly, we know that an attacker can't push arbitrary addresses
        into the shadow stack thanks to the Physical Memory Protection, which
        prevents unauthorized access to the data structure. In this case, the
        tampered instruction is detected and execution is instantly terminated. Note
        that the fact that we can trust the shadow stack is highly dependent on the
        configuration of the Physical Memory Protection. This is because, without
        a proper configuration, it would be possible for an attacker to push a
        value into the shadow stack and then tamper with the return address to
        effectively return to an unauthorized address.
    \end{itemize}

  \item Edge cases: two edge cases may happen during execution. Mainly, we must
    define what happens when we attempt to modify an empty or a full shadow
    stack:
    \begin{itemize}
      \item Push to full shadow stack: when the U-mode \textit{ECALL} handler attempts
        to push an address into a full shadow stack, we prevent the operation
        and terminate execution. This is done because if we can't store such an address,
        we will not be able to provide enough data to determine if the next
        return instruction is legal or not. We would not be able to compare the user-provided
        address with the correct one;

      \item Pop from an empty shadow stack: when the U-mode \textit{ECALL} handler
        attempts to pop an address from an empty shadow stack, we abort the
        operation and terminate execution. This is done because it is impossible
        for a return instruction to be executed before a jump instruction and,
        since this situation represents such an example, the only explanation is
        that the code is trying to perform an unauthorized return instruction.
    \end{itemize}
\end{itemize}

Through the presented Proof of Concept, we successfully demonstrated that the
provided infrastructure can enforce Control Flow Integrity on untrusted user
code, effectively mitigating the risks associated with control flow hijacking attacks.
This capability is vital in maintaining the integrity of execution flow, particularly
when dealing with potentially hazardous code that could be exploited by malicious
actors.

We detailed how both forward and backward edge controls play a crucial role in this
enforcement mechanism. Specifically, forward edge control ensures that jump instructions
only lead to legitimate destinations within the code, while backward edge
control validates return instructions to ensure they correspond to the rightful
call sites. This dual-layered approach fortifies the security framework by establishing
strict legality for these control instructions.

Additionally, we explored the implementation of edge-case controls, which are designed
to anticipate and thwart unexpected behavior that may arise during execution.
These controls add another layer of protection by addressing scenarios that
could potentially bypass traditional control flow checks, thus enhancing the overall
robustness of the system against a wide range of attack vectors.