\chapter{Future Works}
\label{cha:future}

In this thesis, we presented a Control Flow Integrity enforcer that poses itself
as a foundational approach to improving the security of \textit{RISC-V}-based
embedded systems. Moreover, the project showed promising results in both the
security and performance analysis, making it a suitable choice for real-world applications.
However, further advancements can enhance its applicability, efficiency, and robustness.
This chapter outlines potential directions for future work to expand the project's
capabilities.

\section{Additional Security Mechanisms}
\label{sec:future_security}

In previous chapters, we discussed and demonstrated the ability of the
infrastructure to detect and prevent control flow hijacking attacks such as
Return-Oriented Programming or Jump-Oriented Programming. However, in security-demanding
environments, it may be necessary to introduce further techniques to protect the
Control Flow Integrity of the code. Here we list some security mechanisms that
we plan to integrate with future updates.

In section \ref{subsec:background_canaries} we introduced the \textit{Stack
Canary} which is a security mechanism used to detect and prevent stack-based buffer
overflow attacks, a common type of vulnerability in programs. It involves placing
a small, random value in memory just before the stack's return address. This value
acts as a sentinel and is checked for integrity before the function returns control
to its caller. The \textit{Stack Canary} is simple yet effective as it provides a
straightforward way to detect stack corruption caused by buffer overflows.
However, the process of placing and checking canaries adds slight computational and
memory overhead. Overall, we think that introducing a \textit{Stack Canary} in
the project would be a good addition to the security features as it would be
easy to implement and it would provide a slight increase in security.

Since the project suffers from source code modification as we have no way of
telling if the code was modified or not and this could affect the correctness of
forward and backward edge controls we could add a hashing function to the
project. With this function, we could generate a hash of the source code. Note that
we plan to take into account only the ``static'' part of the code and not the code
imported by the user. With this, we could perform a check at each compilation by
simply comparing the initial hash with the freshly generated one. If the hash
differs it means that the source code has been tampered with and should not be
trusted. This would lead to a simple and fast way to determine if the source code
is secure or if it has been modified.

Moreover, many device-specific security measures could be explored as there may
be some devices that provide hardware modules that could be used to further
increase the security of the project. However, this process requires exhaustive
testing and would lead to possible enhancement only on a few embedded devices.

Note that we can't add security measures such as Data Execution Prevention or
Address Space Layout Randomization to the binary as they would conflict with the
system's configuration. Data Execution Prevention requires setting memory regions
with either write but not execute privileges or execute but not write privileges.
This would obviously impact the project as it provides memory regions that
require both privileges. Address Space Layout Randomization instead would randomize
all the addresses so that they are different at each execution. While this
technique provides good security it would render the Control Flow Graph
completely useless as the addresses we collect during the extraction would be
different from the addresses at execution time leading to a situation where each
forward edge control fails.

Lastly, note that the project is designed to protect the code from control flow hijacking
attacks and must be paired with other security measures before deployment. This
is needed in order to ensure security on every attack surface. Otherwise, a
threat actor may be able to exploit other vulnerabilities to perpetrate an
attack bypassing the provided security features.

\section{Code Optimization}
\label{sec:future_optimization}

Although the project has been designed with performance considerations, further
optimizations could enhance its usability in a broader range of embedded
applications. During the performance analysis in chapter \ref{cha:pa} we have seen
that the system comes with an acceptable average time and memory overhead.
However, there are edge cases where this is not true and the overhead grows exponentially.
In future works, we plan to provide optimization solutions to cover such cases lowering
the time and space impact on the execution.

Firstly, we propose a solution to address the space overhead of deeply recursive
algorithms. We have seen that in these cases we need to allocate a very big shadow
stack which increases the size of the produced binary. To solve this, we propose
to add a \textit{peek} function which allows us to look at the first element of
the shadow stack without removing it. When we need to perform a forward edge control
and consequently push the return address into the shadow stack we first check the
top value. If the addresses are the same we avoid pushing the same value more
times. On the other hand, when we need to perform a backward edge control we
look at the first value of the shadow stack. If the two addresses are equal we approve
the return instruction without popping the value. Instead, if the addresses differ,
we pop two values and compare the second one with the return address we are
checking. With this solution, we could effectively address the memory
consumption problem of deeply recursive algorithms without affecting the
security capabilities of the project.

Secondly, we provided a solution to reduce memory usage in cases where the user code
is small. For example, say that the user code starts at address $0x4038A000$ and
ends at address $0x4038F000$. In this case, we can ignore the first $16$ bits of
the address as they do not provide useful information. This means that we could
modify the shadow stack and Control Flow Graph to store \textit{unsigned short} values
instead of \textit{unsigned int} values. Such a solution would halve the memory requirement
for the shadow stack and CFG. Although this solution is very effective in memory
management it is only applicable when the codebase is very small.

Moreover, we have seen that if the user-imported code is very big, take as an
example a firmware, the Control Flow Graph would drastically increase in size.
This is because, the larger the code, the larger the possibility to have many
jump instructions from different source addresses to different destination
addresses. To address this problem, we propose to use a Hash Map instead of a
two-dimensional array to represent the CFG. Although this solution does not reduce
memory usage, it allows for faster lookups as Hash Maps provide accesses in
constant time ($\mathcal{O}(1)$).

Lastly, translating the source code from the \textit{C} programming language
into pure \textit{Assembly} language can significantly enhance performance. This
conversion allows for more fine-grained control over system resources, enabling to
write code that is tailored specifically to the hardware on which it runs. As a
result, it opens up opportunities for advanced optimization techniques that can minimize
both time and memory overhead. By leveraging the specific architecture of the
processor, programmers can streamline execution paths, reduce unnecessary
computations, and manage memory usage more efficiently, which collectively
contributes to a more responsive and efficient application.

\section{Exhaustive Testing}
\label{sec:future_testing}

During the security analysis in chapter \ref{cha:ta} we showcased the effectiveness
of the project in protecting the Control Flow Integrity of the user code.
Moreover, we proved through tests that attempts to perform an unauthorized control
transfer are immediately detected and prevented by the Control Flow Integrity enforcer.
However, we also pointed out that the presented project is not able to mitigate non-control-flow
hijacking attacks. In future works, we plan to perform exhaustive testing of the
project to further inspect its behavior in diverse environments. This could lead
to either stronger proof that the project is actually able to protect the code
or to the discovery of untested scenarios. In the case of the latter, we plan to
provide a solution to eventual unexpected behavior.

\section{RTOS Implementation}
\label{sec:future_rtos}

The main focus for future works sees the integration of a Real-Time Operating System
such as \textit{FreeRTOS} or \textit{Zephyr RTOS}. Such a task is of great
importance since an implementation providing the features of both the project
and an RTOS would drastically increase the number of fields in which this
project could be deployed. This would open the door to more complex solutions and
would allow embedded devices to carry out complex tasks while Control Flow Integrity
is preserved by the project.

We already discussed all the additional features that an RTOS can provide in chapter
\ref{cha:rtos} giving insights on the advantages of using such technology. Also,
we discussed the limitations we would face during the integration of an RTOS.
Note that there are various ways to carry out this task as, depending on the specific
situation, we may want to have the Control Flow Integrity enforcer and the RTOS
running at the same privilege level or, we may want the RTOS to run at an intermediate
level while the CFI enforcer provides security for both the RTOS and the user code.

We aim to enhance the project by providing the following implementation. In our
idea, the Control Flow Integrity enforcer is the only code that runs at M-mode and
it is responsible for managing machine-level traps as well as forward and backward
edge controls. The selected RTOS, be it \textit{FreeRTOS}, \textit{Zephyr RTOS},
or any other RTOS, would run in supervisor mode. This is because we want the RTOS
to be less privileged than the CFI enforcer but more privileged than the user code.
Lastly, the user code would run in U-mode as already described in this thesis.
Figure \ref{fig:rtos} depicts an abstraction of the described implementations
where green, orange, and red colors represent machine, supervisor, and user
modes respectively. \\
\begin{figure}[htbp]
  \centering
  \includegraphics[width=\linewidth]{images/rtos.png}
  \caption{Abstraction of possible RTOS implementations}
  \label{fig:rtos}
\end{figure}

To achieve this result, the first thing we need to do is add the letter \textit{S}
to the Instruction Set Architecture we want to use, making it \textit{RV32IMCS\_ZICSR}
to add supervisor capabilities to our \textit{RISC-V} ISA.

For the second step, we need to import the files required by the RTOS, since many
RTOSes are designed with a modular approach not every file is necessary. For
example, we may want to import only the kernel itself or, if we need networking
capabilities, we would need to import the connectivity-related source files and so
on.

Lastly, we need to modify the instrumentation phase in two ways. Firstly, we
need to add as a target for instrumentation the source code of the RTOS if we
wish to enforce Control Flow Integrity on it. This can be done by simply adding the
files related to the RTOS to the list \textit{files\_to\_instrument} inside the
\textit{flasher.py} file. Secondly, we need to add the regexes and instrumentation
capabilities to delegate CSR instructions the the interrupt vector table of the project.
This is needed if we want the RTOS to run as supervisor since, from such a privilege
level it has no access to Control and Status Registers. A solution to this
implementation is deeply discussed in section \ref{sec:rtos_limitations} where
we discuss each step needed to effectively modify the instrumentation phase to
address the CSR access limitation.

In conclusion, to make the building process easier we may add a simple parameter
to the input of \textit{flasher.py}. As of now, the file can be launched with
the commands described in section \ref{sec:project_instrumentation} but, by simply
adding a parameter like \textit{RTOS=(y/n)} we could launch \textit{flasher.py} with
\textit{python3 flasher.py command rtos}. Such simple modification would make it
easier to either include the RTOS files or not, thus providing a simple and fast
building process to any kind of project.
